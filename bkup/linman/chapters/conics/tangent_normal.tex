\renewcommand{\theequation}{\theenumi}
\begin{enumerate}[label=\thesection.\arabic*.,ref=\thesection.\theenumi]
\numberwithin{equation}{enumi}

\item 
{\em Secant: }The points of intersection of the line 
\begin{align}
L: \quad \vec{x} = \vec{q} + \mu \vec{m} \quad \mu \in \mathbb{R}
\label{eq:conic_tangent}
\end{align}
with the conic section in \eqref{eq:conic_quad_form} are given by
\begin{align}
\vec{x}_i = \vec{q} + \mu_i \vec{m}
\end{align}
%
where
\begin{multline}
\mu_i = \frac{1}
{
\vec{m}^T\vec{V}\vec{m}
}
\lbrak{-\vec{m}^T\brak{\vec{V}\vec{q}+\vec{u}}}
\\
\pm
{\small
\rbrak{\sqrt{
\sbrak{
\vec{m}^T\brak{\vec{V}\vec{q}+\vec{u}}
}^2
-
\brak
{
\vec{q}^T\vec{V}\vec{q} + 2\vec{u}^T\vec{q} +f
}
\brak{\vec{m}^T\vec{V}\vec{m}}
}
}
}
\label{eq:tangent_roots}
\end{multline}
\solution
Substituting \eqref{eq:conic_tangent}
in \eqref{eq:conic_quad_form}, 
\begin{multline}
\brak{\vec{q} + \mu \vec{m}}^T\vec{V}\brak{\vec{q} + \mu \vec{m}}  + 2 \vec{u}^T\brak{\vec{q} + \mu \vec{m}}+f = 0
\\
\implies \mu^2\vec{m}^T\vec{V}\vec{m} + 2 \mu\vec{m}^T\brak{\vec{V}\vec{q}+\vec{u}} 
\\
+ \vec{q}^T\vec{V}\vec{q} + 2\vec{u}^T\vec{q} +f = 0
\label{eq:conic_intercept}
\end{multline}
Solving the above quadratic in \eqref{eq:conic_intercept}
yields \eqref{eq:tangent_roots}.
\item 
{\em Tangent: } If $L$ in \eqref{eq:conic_tangent} touches \eqref{eq:conic_quad_form} at exactly one point $\vec{q}$, 
\begin{align}
\vec{m}^T\brak{\vec{V}\vec{q}+\vec{u}} = 0
\label{eq:conic_tangent_mq}
\end{align}
\solution  In this case, \eqref{eq:conic_intercept} has exactly one root.  Hence, 
in \eqref{eq:tangent_roots}
\begin{multline}
\sbrak{
\vec{m}^T\brak{\vec{V}\vec{q}+\vec{u}}
}^2                                                                                                                    \\
-\brak{\vec{m}^T\vec{V}\vec{m}}
\brak
{
\vec{q}^T\vec{V}\vec{q} + 2\vec{u}^T\vec{q} +f
} = 0                                                                                             
\label{eq:conic_tangent_disc}
\end{multline}                    
$\because \vec{q}$ is the point of contact, $\vec{q}$ satisfies \eqref{eq:conic_quad_form}
and 
\begin{align}
\vec{q}^T\vec{V}\vec{q} + 2\vec{u}^T\vec{q} +f = 0
\label{eq:conic_tangent_qquad}
\end{align}
Substituting \eqref{eq:conic_tangent_qquad} in \eqref{eq:conic_tangent_disc} and simplifying, we obtain \eqref{eq:conic_tangent_mq}.
\item 
The normal vector is obtained from \eqref{eq:conic_tangent_mq} and \eqref{eq:line_dir_norm}
as
%
\begin{align}
\label{eq:conic_normal_vec}
\vec{n} = \vec{V}\vec{q}+\vec{u}
\end{align}
\item 
Given the point of contact $\vec{q}$, the equation of a tangent is 
\begin{align}
\brak{\vec{V}\vec{q}+\vec{u}}^T\vec{x}+\vec{u}^T\vec{q}+f = 0
\label{eq:conic_tangent_final}
\end{align}
\solution
 From \eqref{eq:conic_normal_vec} and \eqref{eq:line_norm_eq}, the equation of the tangent is\begin{align}
\brak{\vec{V}\vec{q}+\vec{u}}^T\brak{\vec{x}-\vec{q}} &=0
\\
\implies \brak{\vec{V}\vec{q}+\vec{u}}^T\vec{x}-\vec{q}^T\vec{V}\vec{q}-\vec{u}^T\vec{q} &= 0
\end{align}
which, upon substituting from \eqref{eq:conic_tangent_qquad} and simplifying yields \eqref{eq:conic_tangent}.
\item 
If $\vec{V}^{-1}$ exists, given the normal vector $\vec{n}$, the tangent points of contact to \eqref{eq:conic_quad_form} are given by
\begin{align}
\vec{q}_i &= \vec{V}^{-1}\brak{\kappa_i \vec{n}-\vec{u}}, i = 1,2
\\
\text{where }\kappa_i &= \pm \sqrt{
\frac{
\vec{u}^T\vec{V}^{-1}\vec{u}-f
}
{
\vec{n}^T\vec{V}^{-1}\vec{n}
}
}
\label{eq:conic_tangent_qk}
\end{align}
\solution  From \eqref{eq:conic_normal_vec},
\begin{align}
\label{eq:conic_normal_vec_q}
 \vec{q} = \vec{V}^{-1}\brak{\kappa \vec{n}-\vec{u}}, \quad \kappa \in \mathbb{R}
\end{align}
Substituting \eqref{eq:conic_normal_vec_q}
in \eqref{eq:conic_tangent_qquad},
\begin{multline}
\brak{\kappa \vec{n}-\vec{u}}^T\vec{V}^{-1}\brak{\kappa \vec{n}-\vec{u}} 
\\
+ 2\vec{u}^T\vec{V}^{-1}\brak{\kappa \vec{n}-\vec{u}} +f = 0
\\
\implies 
\kappa^2 \vec{n}^T\vec{V}^{-1}\vec{n} - \vec{u}^T\vec{V}^{-1}\vec{u} + f =0
\\
\text{or, } \kappa = \pm \sqrt{\frac{\vec{u}^T\vec{V}^{-1}\vec{u}-f}{\vec{n}^T\vec{V}^{-1}\vec{n}}}\label{eq:conic_normal_k}
\end{multline}
%
Substituting \eqref{eq:conic_normal_k} in \eqref{eq:conic_normal_vec_q}
yields \eqref{eq:conic_tangent_qk}.
%
\item 
If $\vec{V}$ is not invertible,  given the normal vector $\vec{n}$, the point of contact to \eqref{eq:conic_quad_form} is given by the matrix equation
\begin{align}
\label{eq:conic_tangent_q_eigen}
\begin{pmatrix}
\vec{u+\kappa \vec{n}}^T \\ \vec{V}
\end{pmatrix}
\vec{q} &= 
\begin{pmatrix}
-f
\\
\kappa\vec{n}-\vec{u}
\end{pmatrix}
\\
\text{where }  \kappa = \frac{\vec{p}_1^T\vec{u}}{\vec{p}_1^T\vec{n}}, \quad \vec{V}\vec{p}_1 &= 0
\label{eq:conic_tangent_qk_eigen}
\end{align}
\solution If $\vec{V}$ is non-invertible, it has a zero eigenvalue.  If the corresponding eigenvector is $\vec{p}_1$, then,
\begin{align}
\vec{V}\vec{p}_1 = 0
\label{eq:conic_zero_eigen}
\end{align}
From \eqref{eq:conic_normal_vec},
\begin{align}
\label{eq:conic_zero_eigen_normal}
\kappa \vec{n} &= \vec{V} \vec{q}+\vec{u}, \quad \kappa \in \mathbb{R}
\\
\implies \kappa \vec{p}_1^T\vec{n} &= \vec{p}_1^T\vec{V} \vec{q}+\vec{p}_1^T\vec{u}
\\
\text{or, } \kappa \vec{p}_1^T\vec{n} &= \vec{p}_1^T\vec{u},  \quad \because \vec{p}_1^T \vec{V} = 0, 
\\
\text{ from } \eqref{eq:conic_zero_eigen} &
%\label{eq:conic_normal_vec_q}
\end{align}
yielding $\kappa$ in \eqref{eq:conic_tangent_qk_eigen}. From \eqref{eq:conic_zero_eigen_normal},
\begin{align}
\kappa \vec{q}^T\vec{n} &= \vec{q}^T\vec{V} \vec{q}+\vec{q}^T\vec{u}
\\
\implies \kappa \vec{q}^T\vec{n} &= -f-\vec{q}^T\vec{u} \quad \text{from } \eqref{eq:conic_tangent_qquad},
\\
\text{or, } \brak{\kappa \vec{n}+\vec{u}}\vec{q} &= -f
\label{eq:conic_zero_eigen_normal_fq}
\end{align}
\eqref{eq:conic_zero_eigen_normal} can be expressed as
\begin{align}
\label{eq:conic_zero_eigen_normal_vq}
\vec{V} \vec{q} = \kappa \vec{n} - \vec{u}.
\end{align}
\eqref{eq:conic_zero_eigen_normal_fq} and \eqref{eq:conic_zero_eigen_normal_vq} clubbed together result in \eqref{eq:conic_tangent_q_eigen}.
\item All the results related to conics are summarized in 
Table \ref{table:conics}.  
\begin{table*}[!t]
\centering
%%%%%%%%%%%%%%%%%%%%%%%%%%%%%%%%%%%%%%%%%%%%%%%%%%%%%%%%%%%%%%%%%%%%%%
%%                                                                  %%
%%  This is the header of a LaTeX2e file exported from Gnumeric.    %%
%%                                                                  %%
%%  This file can be compiled as it stands or included in another   %%
%%  LaTeX document. The table is based on the longtable package so  %%
%%  the longtable options (headers, footers...) can be set in the   %%
%%  preamble section below (see PRAMBLE).                           %%
%%                                                                  %%
%%  To include the file in another, the following two lines must be %%
%%  in the including file:                                          %%
%%        \def\inputGnumericTable{}                                 %%
%%  at the beginning of the file and:                               %%
%%        \input{name-of-this-file.tex}                             %%
%%  where the table is to be placed. Note also that the including   %%
%%  file must use the following packages for the table to be        %%
%%  rendered correctly:                                             %%
%%    \usepackage[latin1]{inputenc}                                 %%
%%    \usepackage{color}                                            %%
%%    \usepackage{array}                                            %%
%%    \usepackage{longtable}                                        %%
%%    \usepackage{calc}                                             %%
%%    \usepackage{multirow}                                         %%
%%    \usepackage{hhline}                                           %%
%%    \usepackage{ifthen}                                           %%
%%  optionally (for landscape tables embedded in another document): %%
%%    \usepackage{lscape}                                           %%
%%                                                                  %%
%%%%%%%%%%%%%%%%%%%%%%%%%%%%%%%%%%%%%%%%%%%%%%%%%%%%%%%%%%%%%%%%%%%%%%



%%  This section checks if we are begin input into another file or  %%
%%  the file will be compiled alone. First use a macro taken from   %%
%%  the TeXbook ex 7.7 (suggestion of Han-Wen Nienhuys).            %%
\def\ifundefined#1{\expandafter\ifx\csname#1\endcsname\relax}


%%  Check for the \def token for inputed files. If it is not        %%
%%  defined, the file will be processed as a standalone and the     %%
%%  preamble will be used.                                          %%
\ifundefined{inputGnumericTable}

%%  We must be able to close or not the document at the end.        %%
	\def\gnumericTableEnd{\end{document}}


%%%%%%%%%%%%%%%%%%%%%%%%%%%%%%%%%%%%%%%%%%%%%%%%%%%%%%%%%%%%%%%%%%%%%%
%%                                                                  %%
%%  This is the PREAMBLE. Change these values to get the right      %%
%%  paper size and other niceties.                                  %%
%%                                                                  %%
%%%%%%%%%%%%%%%%%%%%%%%%%%%%%%%%%%%%%%%%%%%%%%%%%%%%%%%%%%%%%%%%%%%%%%

	\documentclass[12pt%
			  %,landscape%
                    ]{report}
       \usepackage[latin1]{inputenc}
       \usepackage{fullpage}
       \usepackage{color}
       \usepackage{array}
       \usepackage{longtable}
       \usepackage{calc}
       \usepackage{multirow}
       \usepackage{hhline}
       \usepackage{ifthen}

	\begin{document}


%%  End of the preamble for the standalone. The next section is for %%
%%  documents which are included into other LaTeX2e files.          %%
\else

%%  We are not a stand alone document. For a regular table, we will %%
%%  have no preamble and only define the closing to mean nothing.   %%
    \def\gnumericTableEnd{}

%%  If we want landscape mode in an embedded document, comment out  %%
%%  the line above and uncomment the two below. The table will      %%
%%  begin on a new page and run in landscape mode.                  %%
%       \def\gnumericTableEnd{\end{landscape}}
%       \begin{landscape}


%%  End of the else clause for this file being \input.              %%
\fi

%%%%%%%%%%%%%%%%%%%%%%%%%%%%%%%%%%%%%%%%%%%%%%%%%%%%%%%%%%%%%%%%%%%%%%
%%                                                                  %%
%%  The rest is the gnumeric table, except for the closing          %%
%%  statement. Changes below will alter the table's appearance.     %%
%%                                                                  %%
%%%%%%%%%%%%%%%%%%%%%%%%%%%%%%%%%%%%%%%%%%%%%%%%%%%%%%%%%%%%%%%%%%%%%%

\providecommand{\gnumericmathit}[1]{#1} 
%%  Uncomment the next line if you would like your numbers to be in %%
%%  italics if they are italizised in the gnumeric table.           %%
%\renewcommand{\gnumericmathit}[1]{\mathit{#1}}
\providecommand{\gnumericPB}[1]%
{\let\gnumericTemp=\\#1\let\\=\gnumericTemp\hspace{0pt}}
 \ifundefined{gnumericTableWidthDefined}
        \newlength{\gnumericTableWidth}
        \newlength{\gnumericTableWidthComplete}
        \newlength{\gnumericMultiRowLength}
        \global\def\gnumericTableWidthDefined{}
 \fi
%% The following setting protects this code from babel shorthands.  %%
 \ifthenelse{\isundefined{\languageshorthands}}{}{\languageshorthands{english}}
%%  The default table format retains the relative column widths of  %%
%%  gnumeric. They can easily be changed to c, r or l. In that case %%
%%  you may want to comment out the next line and uncomment the one %%
%%  thereafter                                                      %%
\providecommand\gnumbox{\makebox[0pt]}
%%\providecommand\gnumbox[1][]{\makebox}

%% to adjust positions in multirow situations                       %%
\setlength{\bigstrutjot}{\jot}
\setlength{\extrarowheight}{\doublerulesep}

%%  The \setlongtables command keeps column widths the same across  %%
%%  pages. Simply comment out next line for varying column widths.  %%
\setlongtables

\setlength\gnumericTableWidth{%
	59pt+%
	50pt+%
	70pt+%
	76pt+%
	67pt+%
0pt}
\def\gumericNumCols{5}
\setlength\gnumericTableWidthComplete{\gnumericTableWidth+%
         \tabcolsep*\gumericNumCols*2+\arrayrulewidth*\gumericNumCols}
\ifthenelse{\lengthtest{\gnumericTableWidthComplete > \linewidth}}%
         {\def\gnumericScale{\ratio{\linewidth-%
                        \tabcolsep*\gumericNumCols*2-%
                        \arrayrulewidth*\gumericNumCols}%
{\gnumericTableWidth}}}%
{\def\gnumericScale{1}}

%%%%%%%%%%%%%%%%%%%%%%%%%%%%%%%%%%%%%%%%%%%%%%%%%%%%%%%%%%%%%%%%%%%%%%
%%                                                                  %%
%% The following are the widths of the various columns. We are      %%
%% defining them here because then they are easier to change.       %%
%% Depending on the cell formats we may use them more than once.    %%
%%                                                                  %%
%%%%%%%%%%%%%%%%%%%%%%%%%%%%%%%%%%%%%%%%%%%%%%%%%%%%%%%%%%%%%%%%%%%%%%

\ifthenelse{\isundefined{\gnumericColA}}{\newlength{\gnumericColA}}{}\settowidth{\gnumericColA}{\begin{tabular}{@{}p{59pt*\gnumericScale}@{}}x\end{tabular}}
\ifthenelse{\isundefined{\gnumericColB}}{\newlength{\gnumericColB}}{}\settowidth{\gnumericColB}{\begin{tabular}{@{}p{50pt*\gnumericScale}@{}}x\end{tabular}}
\ifthenelse{\isundefined{\gnumericColC}}{\newlength{\gnumericColC}}{}\settowidth{\gnumericColC}{\begin{tabular}{@{}p{70pt*\gnumericScale}@{}}x\end{tabular}}
\ifthenelse{\isundefined{\gnumericColD}}{\newlength{\gnumericColD}}{}\settowidth{\gnumericColD}{\begin{tabular}{@{}p{76pt*\gnumericScale}@{}}x\end{tabular}}
\ifthenelse{\isundefined{\gnumericColE}}{\newlength{\gnumericColE}}{}\settowidth{\gnumericColE}{\begin{tabular}{@{}p{67pt*\gnumericScale}@{}}x\end{tabular}}

{\footnotesize
\begin{tabular}[c]{%
	b{\gnumericColA}%
	b{\gnumericColB}%
	b{\gnumericColC}%
	b{\gnumericColD}%
	b{\gnumericColE}%
	}

%%%%%%%%%%%%%%%%%%%%%%%%%%%%%%%%%%%%%%%%%%%%%%%%%%%%%%%%%%%%%%%%%%%%%%
%%  The longtable options. (Caption, headers... see Goosens, p.124) %%
%	\caption{The Table Caption.}             \\	%
% \hline	% Across the top of the table.
%%  The rest of these options are table rows which are placed on    %%
%%  the first, last or every page. Use \multicolumn if you want.    %%

%%  Header for the first page.                                      %%
%	\multicolumn{5}{c}{The First Header} \\ \hline 
%	\multicolumn{1}{c}{colTag}	%Column 1
%	&\multicolumn{1}{c}{colTag}	%Column 2
%	&\multicolumn{1}{c}{colTag}	%Column 3
%	&\multicolumn{1}{c}{colTag}	%Column 4
%	&\multicolumn{1}{c}{colTag}	\\ \hline %Last column
%	\endfirsthead

%%  The running header definition.                                  %%
%	\hline
%	\multicolumn{5}{l}{\ldots\small\slshape continued} \\ \hline
%	\multicolumn{1}{c}{colTag}	%Column 1
%	&\multicolumn{1}{c}{colTag}	%Column 2
%	&\multicolumn{1}{c}{colTag}	%Column 3
%	&\multicolumn{1}{c}{colTag}	%Column 4
%	&\multicolumn{1}{c}{colTag}	\\ \hline %Last column
%	\endhead

%%  The running footer definition.                                  %%
%	\hline
%	\multicolumn{5}{r}{\small\slshape continued\ldots} \\
%	\endfoot

%%  The ending footer definition.                                   %%
%	\multicolumn{5}{c}{That's all folks} \\ \hline 
%	\endlastfoot
%%%%%%%%%%%%%%%%%%%%%%%%%%%%%%%%%%%%%%%%%%%%%%%%%%%%%%%%%%%%%%%%%%%%%%

\hhline{|-|-|-|-|-}
	 \multicolumn{1}{|p{\gnumericColA}|}%
	{\gnumericPB{\centering}\textbf{Conic}}
	&\multicolumn{1}{p{\gnumericColB}|}%
	{\gnumericPB{\centering}\textbf{Property}}
	&\multicolumn{1}{p{\gnumericColC}|}%
	{\gnumericPB{\centering}\textbf{Standard Form}}
	&\multicolumn{1}{p{\gnumericColD}|}%
	{\gnumericPB{\centering}\textbf{Standard Parameters}}
	&\multicolumn{1}{p{\gnumericColE}|}%
	{\gnumericPB{\centering}\textbf{Point(s) of Contact}}
\\
\hhline{|-----|}
	 \multicolumn{1}{|p{\gnumericColA}|}%
	{\gnumericPB{\centering}\textbf{Circle}}
	&\multicolumn{1}{p{\gnumericColB}|}%
	{\gnumericPB{\centering}$\vec{V} = \vec{I}$}
	&\multicolumn{1}{p{\gnumericColC}|}%
	{\setlength{\gnumericMultiRowLength}{0pt}%
	 \addtolength{\gnumericMultiRowLength}{\gnumericColC}%
	 \multirow{3}[1]{\gnumericMultiRowLength}{\parbox{\gnumericMultiRowLength}{%
	\scriptsize \gnumericPB{\centering}$\begin{array}{c}\frac{\vec{y}^T\vec{D}\vec{y}}{\vec{u}^T\vec{V}^{-1}\vec{u}-f} = 1
\\
\vec{D} = \cmyvec{\lambda_1 & 0 \\ 0 & \lambda_2}
\\
\vec{V} = \vec{P}\vec{D}\vec{P}^T
\\
\vec{P} = \myvec{\vec{p}_1 & \vec{p}_2}
\end{array}$}}}
	&\multicolumn{1}{p{\gnumericColD}|}%
	{\scriptsize \gnumericPB{\centering}$\begin{array}{c}\vec{c} = -\vec{u}, \\ r = \sqrt{\vec{u}^T\vec{u}-f}\end{array}$}
	&\multicolumn{1}{p{\gnumericColE}|}%
	{\setlength{\gnumericMultiRowLength}{0pt}%
	 \addtolength{\gnumericMultiRowLength}{\gnumericColE}%
	 \multirow{3}[1]{\gnumericMultiRowLength}{\parbox{\gnumericMultiRowLength}{%
\tiny	 \gnumericPB{\centering}\begin{multline*} \vec{q} = \vec{V}^{-1}\brak{\kappa \vec{n}-\vec{u}}
\\
\kappa = \pm \sqrt{\frac{\vec{u}^T\vec{V}^{-1}\vec{u}-f}{\vec{n}^T\vec{V}^{-1}\vec{n}}}
\end{multline*}}}}
\\
\hhline{|--|~|-|~}
	 \multicolumn{1}{|p{\gnumericColA}|}%
	{\gnumericPB{\centering}\textbf{Ellipse}}
	&\multicolumn{1}{p{\gnumericColB}|}%
	{\gnumericPB{\centering}$\begin{array}{c}\abs{\vec{V}} > 0 \\ \lambda_1 > 0, \lambda_2 < 0 \end{array}$}
	&\multicolumn{1}{p{\gnumericColC}|}%
	{}
	&\multicolumn{1}{p{\gnumericColD}|}%
	{\tiny \gnumericPB{\centering}$\begin{array}{c}\vec{c} = -\vec{V}^{-1}\vec{u},\\ 
axes = 
\begin{cases}\sqrt{\frac{\vec{u}^T\vec{V}^{-1}\vec{u}-f}{\lambda_1}} \\ \sqrt{\frac{\vec{u}^T\vec{V}^{-1}\vec{u}-f}{\lambda_2}} \end{cases}
\end{array}$}
	&\multicolumn{1}{p{\gnumericColE}|}%
	{}
\\
\hhline{|--|~|-|~}
	 \multicolumn{1}{|p{\gnumericColA}|}%
	{\gnumericPB{\centering}\textbf{Hyperbola}}
	&\multicolumn{1}{p{\gnumericColB}|}%
	{\gnumericPB{\centering}$\begin{array}{c}\abs{\vec{V}} < 0\\ \lambda_1 > 0, \lambda_2 < 0\end{array}$}
	&\multicolumn{1}{p{\gnumericColC}|}%
	{}
	&\multicolumn{1}{p{\gnumericColD}|}%
	{\tiny \gnumericPB{\centering}
$\begin{array}{c}\vec{c} = -\vec{V}^{-1}\vec{u}, \\ 
axes = 
\begin{cases}
\sqrt{\frac{\vec{u}^T\vec{V}^{-1}\vec{u}-f}{\lambda_1}}\\ \sqrt{\frac{f-\vec{u}^T\vec{V}^{-1}\vec{u}}{\lambda_2}}
\end{cases}
\end{array}$}
	&\multicolumn{1}{p{\gnumericColE}|}%
	{}
\\
\hhline{|-----|}
	 \multicolumn{1}{|p{\gnumericColA}|}%
	{\gnumericPB{\centering}\textbf{Parabola}}
	&\multicolumn{1}{p{\gnumericColB}|}%
	{\gnumericPB{\centering}$\begin{array}{c}\abs{\vec{V}} = 0 \\ \lambda_1 = 0 \end{array}$}
	&\multicolumn{1}{p{\gnumericColC}|}%
	{\tiny  $\vec{y}^T\vec{D}\vec{y} =  -2\eta\myvec{1 & 0}\vec{y}$}
	&\multicolumn{1}{p{\gnumericColD}|}%
	{\tiny \gnumericPB{\centering}{ \begin{multline*} 
\text{focal length} = \mydet{\frac{2\eta}{\lambda_2}}
\\ 
\cmyvec{ \vec{u}^T+\eta\vec{p}_1^T \\ \vec{V}}\vec{c} 
\\
= \cmyvec{-f \\ \eta\vec{p}_1-\vec{u}}
\\
\eta = \vec{p}_1^T\vec{u}
\end{multline*}}}
	&\multicolumn{1}{p{\gnumericColE}|}%
	{\tiny \gnumericPB{\centering}{ \begin{multline*}
\cmyvec{ \vec{u+\kappa \vec{n}}^T \\ \vec{V}}\vec{q} 
\\
= \cmyvec{-f
\\
\kappa\vec{n}-\vec{u}}
\\
\kappa = \frac{\vec{p_1}^T\vec{u}}{\vec{p_1}^T\vec{n}}
\end{multline*}}}
\\
\hhline{|-|-|-|-|-|}
\end{tabular}
}
\ifthenelse{\isundefined{\languageshorthands}}{}{\languageshorthands{\languagename}}
\gnumericTableEnd

%%%%%%%%%%%%%%%%%%%%%%%%%%%%%%%%%%%%%%%%%%%%%%%%%%%%%%%%%%%%%%%%%%%%%%%
%%                                                                  %%
%%  This is the header of a LaTeX2e file exported from Gnumeric.    %%
%%                                                                  %%
%%  This file can be compiled as it stands or included in another   %%
%%  LaTeX document. The table is based on the longtable package so  %%
%%  the longtable options (headers, footers...) can be set in the   %%
%%  preamble section below (see PRAMBLE).                           %%
%%                                                                  %%
%%  To include the file in another, the following two lines must be %%
%%  in the including file:                                          %%
%%        \def\inputGnumericTable{}                                 %%
%%  at the beginning of the file and:                               %%
%%        \input{name-of-this-file.tex}                             %%
%%  where the table is to be placed. Note also that the including   %%
%%  file must use the following packages for the table to be        %%
%%  rendered correctly:                                             %%
%%    \usepackage[latin1]{inputenc}                                 %%
%%    \usepackage{color}                                            %%
%%    \usepackage{array}                                            %%
%%    \usepackage{longtable}                                        %%
%%    \usepackage{calc}                                             %%
%%    \usepackage{multirow}                                         %%
%%    \usepackage{hhline}                                           %%
%%    \usepackage{ifthen}                                           %%
%%  optionally (for landscape tables embedded in another document): %%
%%    \usepackage{lscape}                                           %%
%%                                                                  %%
%%%%%%%%%%%%%%%%%%%%%%%%%%%%%%%%%%%%%%%%%%%%%%%%%%%%%%%%%%%%%%%%%%%%%%



%%  This section checks if we are begin input into another file or  %%
%%  the file will be compiled alone. First use a macro taken from   %%
%%  the TeXbook ex 7.7 (suggestion of Han-Wen Nienhuys).            %%
\def\ifundefined#1{\expandafter\ifx\csname#1\endcsname\relax}


%%  Check for the \def token for inputed files. If it is not        %%
%%  defined, the file will be processed as a standalone and the     %%
%%  preamble will be used.                                          %%
\ifundefined{inputGnumericTable}

%%  We must be able to close or not the document at the end.        %%
	\def\gnumericTableEnd{\end{document}}


%%%%%%%%%%%%%%%%%%%%%%%%%%%%%%%%%%%%%%%%%%%%%%%%%%%%%%%%%%%%%%%%%%%%%%
%%                                                                  %%
%%  This is the PREAMBLE. Change these values to get the right      %%
%%  paper size and other niceties.                                  %%
%%                                                                  %%
%%%%%%%%%%%%%%%%%%%%%%%%%%%%%%%%%%%%%%%%%%%%%%%%%%%%%%%%%%%%%%%%%%%%%%

	\documentclass[12pt%
			  %,landscape%
                    ]{report}
       \usepackage[latin1]{inputenc}
       \usepackage{fullpage}
       \usepackage{color}
       \usepackage{array}
       \usepackage{longtable}
       \usepackage{calc}
       \usepackage{multirow}
       \usepackage{hhline}
       \usepackage{ifthen}

	\begin{document}


%%  End of the preamble for the standalone. The next section is for %%
%%  documents which are included into other LaTeX2e files.          %%
\else

%%  We are not a stand alone document. For a regular table, we will %%
%%  have no preamble and only define the closing to mean nothing.   %%
    \def\gnumericTableEnd{}

%%  If we want landscape mode in an embedded document, comment out  %%
%%  the line above and uncomment the two below. The table will      %%
%%  begin on a new page and run in landscape mode.                  %%
%       \def\gnumericTableEnd{\end{landscape}}
%       \begin{landscape}


%%  End of the else clause for this file being \input.              %%
\fi

%%%%%%%%%%%%%%%%%%%%%%%%%%%%%%%%%%%%%%%%%%%%%%%%%%%%%%%%%%%%%%%%%%%%%%
%%                                                                  %%
%%  The rest is the gnumeric table, except for the closing          %%
%%  statement. Changes below will alter the table's appearance.     %%
%%                                                                  %%
%%%%%%%%%%%%%%%%%%%%%%%%%%%%%%%%%%%%%%%%%%%%%%%%%%%%%%%%%%%%%%%%%%%%%%

\providecommand{\gnumericmathit}[1]{#1} 
%%  Uncomment the next line if you would like your numbers to be in %%
%%  italics if they are italizised in the gnumeric table.           %%
%\renewcommand{\gnumericmathit}[1]{\mathit{#1}}
\providecommand{\gnumericPB}[1]%
{\let\gnumericTemp=\\#1\let\\=\gnumericTemp\hspace{0pt}}
 \ifundefined{gnumericTableWidthDefined}
        \newlength{\gnumericTableWidth}
        \newlength{\gnumericTableWidthComplete}
        \newlength{\gnumericMultiRowLength}
        \global\def\gnumericTableWidthDefined{}
 \fi
%% The following setting protects this code from babel shorthands.  %%
 \ifthenelse{\isundefined{\languageshorthands}}{}{\languageshorthands{english}}
%%  The default table format retains the relative column widths of  %%
%%  gnumeric. They can easily be changed to c, r or l. In that case %%
%%  you may want to comment out the next line and uncomment the one %%
%%  thereafter                                                      %%
\providecommand\gnumbox{\makebox[0pt]}
%%\providecommand\gnumbox[1][]{\makebox}

%% to adjust positions in multirow situations                       %%
\setlength{\bigstrutjot}{\jot}
\setlength{\extrarowheight}{\doublerulesep}

%%  The \setlongtables command keeps column widths the same across  %%
%%  pages. Simply comment out next line for varying column widths.  %%
\setlongtables

\setlength\gnumericTableWidth{%
	59pt+%
	50pt+%
	70pt+%
	76pt+%
	67pt+%
0pt}
\def\gumericNumCols{5}
\setlength\gnumericTableWidthComplete{\gnumericTableWidth+%
         \tabcolsep*\gumericNumCols*2+\arrayrulewidth*\gumericNumCols}
\ifthenelse{\lengthtest{\gnumericTableWidthComplete > \linewidth}}%
         {\def\gnumericScale{\ratio{\linewidth-%
                        \tabcolsep*\gumericNumCols*2-%
                        \arrayrulewidth*\gumericNumCols}%
{\gnumericTableWidth}}}%
{\def\gnumericScale{1}}

%%%%%%%%%%%%%%%%%%%%%%%%%%%%%%%%%%%%%%%%%%%%%%%%%%%%%%%%%%%%%%%%%%%%%%
%%                                                                  %%
%% The following are the widths of the various columns. We are      %%
%% defining them here because then they are easier to change.       %%
%% Depending on the cell formats we may use them more than once.    %%
%%                                                                  %%
%%%%%%%%%%%%%%%%%%%%%%%%%%%%%%%%%%%%%%%%%%%%%%%%%%%%%%%%%%%%%%%%%%%%%%

\ifthenelse{\isundefined{\gnumericColA}}{\newlength{\gnumericColA}}{}\settowidth{\gnumericColA}{\begin{tabular}{@{}p{59pt*\gnumericScale}@{}}x\end{tabular}}
\ifthenelse{\isundefined{\gnumericColB}}{\newlength{\gnumericColB}}{}\settowidth{\gnumericColB}{\begin{tabular}{@{}p{50pt*\gnumericScale}@{}}x\end{tabular}}
\ifthenelse{\isundefined{\gnumericColC}}{\newlength{\gnumericColC}}{}\settowidth{\gnumericColC}{\begin{tabular}{@{}p{70pt*\gnumericScale}@{}}x\end{tabular}}
\ifthenelse{\isundefined{\gnumericColD}}{\newlength{\gnumericColD}}{}\settowidth{\gnumericColD}{\begin{tabular}{@{}p{76pt*\gnumericScale}@{}}x\end{tabular}}
\ifthenelse{\isundefined{\gnumericColE}}{\newlength{\gnumericColE}}{}\settowidth{\gnumericColE}{\begin{tabular}{@{}p{67pt*\gnumericScale}@{}}x\end{tabular}}

{\footnotesize
\begin{tabular}[c]{%
	b{\gnumericColA}%
	b{\gnumericColB}%
	b{\gnumericColC}%
	b{\gnumericColD}%
	b{\gnumericColE}%
	}

%%%%%%%%%%%%%%%%%%%%%%%%%%%%%%%%%%%%%%%%%%%%%%%%%%%%%%%%%%%%%%%%%%%%%%
%%  The longtable options. (Caption, headers... see Goosens, p.124) %%
%	\caption{The Table Caption.}             \\	%
% \hline	% Across the top of the table.
%%  The rest of these options are table rows which are placed on    %%
%%  the first, last or every page. Use \multicolumn if you want.    %%

%%  Header for the first page.                                      %%
%	\multicolumn{5}{c}{The First Header} \\ \hline 
%	\multicolumn{1}{c}{colTag}	%Column 1
%	&\multicolumn{1}{c}{colTag}	%Column 2
%	&\multicolumn{1}{c}{colTag}	%Column 3
%	&\multicolumn{1}{c}{colTag}	%Column 4
%	&\multicolumn{1}{c}{colTag}	\\ \hline %Last column
%	\endfirsthead

%%  The running header definition.                                  %%
%	\hline
%	\multicolumn{5}{l}{\ldots\small\slshape continued} \\ \hline
%	\multicolumn{1}{c}{colTag}	%Column 1
%	&\multicolumn{1}{c}{colTag}	%Column 2
%	&\multicolumn{1}{c}{colTag}	%Column 3
%	&\multicolumn{1}{c}{colTag}	%Column 4
%	&\multicolumn{1}{c}{colTag}	\\ \hline %Last column
%	\endhead

%%  The running footer definition.                                  %%
%	\hline
%	\multicolumn{5}{r}{\small\slshape continued\ldots} \\
%	\endfoot

%%  The ending footer definition.                                   %%
%	\multicolumn{5}{c}{That's all folks} \\ \hline 
%	\endlastfoot
%%%%%%%%%%%%%%%%%%%%%%%%%%%%%%%%%%%%%%%%%%%%%%%%%%%%%%%%%%%%%%%%%%%%%%

\hhline{|-|-|-|-|-}
	 \multicolumn{1}{|p{\gnumericColA}|}%
	{\gnumericPB{\centering}\textbf{Conic}}
	&\multicolumn{1}{p{\gnumericColB}|}%
	{\gnumericPB{\centering}\textbf{Property}}
	&\multicolumn{1}{p{\gnumericColC}|}%
	{\gnumericPB{\centering}\textbf{Standard Form}}
	&\multicolumn{1}{p{\gnumericColD}|}%
	{\gnumericPB{\centering}\textbf{Standard Parameters}}
	&\multicolumn{1}{p{\gnumericColE}|}%
	{\gnumericPB{\centering}\textbf{Point(s) of Contact}}
\\
\hhline{|-----|}
	 \multicolumn{1}{|p{\gnumericColA}|}%
	{\gnumericPB{\centering}\textbf{Circle}}
	&\multicolumn{1}{p{\gnumericColB}|}%
	{\gnumericPB{\centering}$\vec{V} = \vec{I}$}
	&\multicolumn{1}{p{\gnumericColC}|}%
	{\setlength{\gnumericMultiRowLength}{0pt}%
	 \addtolength{\gnumericMultiRowLength}{\gnumericColC}%
	 \multirow{3}[1]{\gnumericMultiRowLength}{\parbox{\gnumericMultiRowLength}{%
	\scriptsize \gnumericPB{\centering}$\begin{array}{c}\frac{\vec{y}^T\vec{D}\vec{y}}{\vec{u}^T\vec{V}^{-1}\vec{u}-f} = 1
\\
\vec{D} = \cmyvec{\lambda_1 & 0 \\ 0 & \lambda_2}
\\
\vec{V} = \vec{P}\vec{D}\vec{P}^T
\\
\vec{P} = \myvec{\vec{p}_1 & \vec{p}_2}
\end{array}$}}}
	&\multicolumn{1}{p{\gnumericColD}|}%
	{\scriptsize \gnumericPB{\centering}$\begin{array}{c}\vec{c} = -\vec{u}, \\ r = \sqrt{\vec{u}^T\vec{u}-f}\end{array}$}
	&\multicolumn{1}{p{\gnumericColE}|}%
	{\setlength{\gnumericMultiRowLength}{0pt}%
	 \addtolength{\gnumericMultiRowLength}{\gnumericColE}%
	 \multirow{3}[1]{\gnumericMultiRowLength}{\parbox{\gnumericMultiRowLength}{%
\tiny	 \gnumericPB{\centering}\begin{multline*} \vec{q} = \vec{V}^{-1}\brak{\kappa \vec{n}-\vec{u}}
\\
\kappa = \pm \sqrt{\frac{\vec{u}^T\vec{V}^{-1}\vec{u}-f}{\vec{n}^T\vec{V}^{-1}\vec{n}}}
\end{multline*}}}}
\\
\hhline{|--|~|-|~}
	 \multicolumn{1}{|p{\gnumericColA}|}%
	{\gnumericPB{\centering}\textbf{Ellipse}}
	&\multicolumn{1}{p{\gnumericColB}|}%
	{\gnumericPB{\centering}$\begin{array}{c}\abs{\vec{V}} > 0 \\ \lambda_1 > 0, \lambda_2 < 0 \end{array}$}
	&\multicolumn{1}{p{\gnumericColC}|}%
	{}
	&\multicolumn{1}{p{\gnumericColD}|}%
	{\tiny \gnumericPB{\centering}$\begin{array}{c}\vec{c} = -\vec{V}^{-1}\vec{u},\\ 
axes = 
\begin{cases}\sqrt{\frac{\vec{u}^T\vec{V}^{-1}\vec{u}-f}{\lambda_1}} \\ \sqrt{\frac{\vec{u}^T\vec{V}^{-1}\vec{u}-f}{\lambda_2}} \end{cases}
\end{array}$}
	&\multicolumn{1}{p{\gnumericColE}|}%
	{}
\\
\hhline{|--|~|-|~}
	 \multicolumn{1}{|p{\gnumericColA}|}%
	{\gnumericPB{\centering}\textbf{Hyperbola}}
	&\multicolumn{1}{p{\gnumericColB}|}%
	{\gnumericPB{\centering}$\begin{array}{c}\abs{\vec{V}} < 0\\ \lambda_1 > 0, \lambda_2 < 0\end{array}$}
	&\multicolumn{1}{p{\gnumericColC}|}%
	{}
	&\multicolumn{1}{p{\gnumericColD}|}%
	{\tiny \gnumericPB{\centering}
$\begin{array}{c}\vec{c} = -\vec{V}^{-1}\vec{u}, \\ 
axes = 
\begin{cases}
\sqrt{\frac{\vec{u}^T\vec{V}^{-1}\vec{u}-f}{\lambda_1}}\\ \sqrt{\frac{f-\vec{u}^T\vec{V}^{-1}\vec{u}}{\lambda_2}}
\end{cases}
\end{array}$}
	&\multicolumn{1}{p{\gnumericColE}|}%
	{}
\\
\hhline{|-----|}
	 \multicolumn{1}{|p{\gnumericColA}|}%
	{\gnumericPB{\centering}\textbf{Parabola}}
	&\multicolumn{1}{p{\gnumericColB}|}%
	{\gnumericPB{\centering}$\begin{array}{c}\abs{\vec{V}} = 0 \\ \lambda_1 = 0 \end{array}$}
	&\multicolumn{1}{p{\gnumericColC}|}%
	{\tiny  $\vec{y}^T\vec{D}\vec{y} =  -2\eta\myvec{1 & 0}\vec{y}$}
	&\multicolumn{1}{p{\gnumericColD}|}%
	{\tiny \gnumericPB{\centering}{ \begin{multline*} 
\text{focal length} = \mydet{\frac{2\eta}{\lambda_2}}
\\ 
\cmyvec{ \vec{u}^T+\eta\vec{p}_1^T \\ \vec{V}}\vec{c} 
\\
= \cmyvec{-f \\ \eta\vec{p}_1-\vec{u}}
\\
\eta = \vec{p}_1^T\vec{u}
\end{multline*}}}
	&\multicolumn{1}{p{\gnumericColE}|}%
	{\tiny \gnumericPB{\centering}{ \begin{multline*}
\cmyvec{ \vec{u+\kappa \vec{n}}^T \\ \vec{V}}\vec{q} 
\\
= \cmyvec{-f
\\
\kappa\vec{n}-\vec{u}}
\\
\kappa = \frac{\vec{p_1}^T\vec{u}}{\vec{p_1}^T\vec{n}}
\end{multline*}}}
\\
\hhline{|-|-|-|-|-|}
\end{tabular}
}
\ifthenelse{\isundefined{\languageshorthands}}{}{\languageshorthands{\languagename}}
\gnumericTableEnd

\caption{$\vec{x}^T\vec{V}\vec{x}+2\vec{u}^T\vec{x}+f = 0$  can be expressed in the above standard form for various conics. $\vec{c}$ represents the centre/vertex of the conic. $\vec{q}$ is/are the point(s) of contact for the tangent(s). }
\label{table:conics}
\end{table*}

\end{enumerate}
