\iffalse
\documentclass[10pt]{article}
       \usepackage[latin1]{inputenc}
       \usepackage{fullpage}
       \usepackage{color}
       \usepackage{array}
       \usepackage{longtable}
       \usepackage{calc}
       \usepackage{multirow}
       \usepackage{hhline}
       \usepackage{ifthen}
\usepackage{graphicx}
\def\inputGnumericTable{}
\usepackage[none]{hyphenat}
\usepackage{graphicx}
\usepackage{listings}
\usepackage[english]{babel}
\usepackage{graphicx}
\usepackage{caption} 
\usepackage{booktabs}
\usepackage{gensymb}
\usepackage{array}
\usepackage{amssymb} % for \because
\usepackage{amsmath}   % for having text in math mode
\usepackage{extarrows} % for Row operations arrows
\usepackage{listings}
\lstset{
  frame=single,
  breaklines=true
}
\usepackage{hyperref}
%Following 2 lines were added to remove the blank page at the beginning
\usepackage{atbegshi}% http://ctan.org/pkg/atbegshi
\AtBeginDocument{\AtBeginShipoutNext{\AtBeginShipoutDiscard}}
%New macro definitions
\newcommand{\mydet}[1]{\ensuremath{\begin{vmatrix}#1\end{vmatrix}}}
\providecommand{\brak}[1]{\ensuremath{\left(#1\right)}}
\providecommand{\norm}[1]{\left\lVert#1\right\rVert}
\newcommand{\solution}{\noindent \textbf{Solution: }}
\newcommand{\myvec}[1]{\ensuremath{\begin{pmatrix}#1\end{pmatrix}}}
\providecommand{\abs}[1]{\left\vert#1\right\vert}
\let\vec\mathbf
\begin{document}

\begin{center}
\title{\textbf{STRAIGHT LINES}}
\date{\vspace{-5ex}} %Not to print date automatically
\maketitle
\end{center}

\section{11$^{th}$ Maths - Chapter 10}
This is Problem 5 from Exercise-10.4
\begin{enumerate}

\solution
\fi
Let
\begin{align}
	\vec{A}=\myvec{\cos\theta\\\sin\theta},\vec{B}&=\myvec{\cos\phi\\\sin\phi}\\
\implies	\vec{m}=\vec{B}-\vec{A}&=\myvec{\cos\phi-\cos\theta\\\sin\phi-\sin\theta}
\end{align}
The normal vector is then given by,
\begin{align}
\vec{n}=\myvec{\sin\phi-\sin\theta\\\cos\theta-\cos\phi} \implies
\norm{\vec{n}}=2\sin\brak{\frac{\phi-\theta}{2}}
\end{align}
The equation of the line is
\begin{align}
\vec{n}^\top\brak{\vec{x}-\vec{A}}&=0\\
\implies\myvec{\sin\phi-\sin\theta&\cos\theta-\cos\phi}\vec{x}&=\sin\brak{\phi-\theta}
\label{eq:chapters/11/10/4/5/1}
\end{align}
Thus, 
\begin{align}
c=\sin\brak{\phi-\theta}
\end{align}
The perpendicular distance from the origin to the line is
\begin{align}
d&=\frac{\abs{c}}{\norm{\vec{n}}}\\
\implies d&=\frac{\sin\brak{\phi-\theta}}{2\sin\brak{\frac{\phi-\theta}{2}}} = \cos\brak{\frac{\phi-\theta}{2}}
\label{eq:chapters/11/10/4/5/2}
\end{align}
