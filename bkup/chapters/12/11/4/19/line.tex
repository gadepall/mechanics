\iffalse
\documentclass[journal,12pt,twocolumn]{IEEEtran}
\usepackage{setspace}
\usepackage{gensymb}
\usepackage{xcolor}
\usepackage{caption}
\singlespacing
\usepackage{siunitx}
\usepackage[cmex10]{amsmath}
\usepackage{mathtools}
\usepackage{hyperref}
\usepackage{amsthm}
\usepackage{mathrsfs}
\usepackage{txfonts}
\usepackage{stfloats}
\usepackage{cite}
\usepackage{cases}
\usepackage{subfig}
\usepackage{longtable}
\usepackage{multirow}
\usepackage{enumitem}
\usepackage{bm}
\usepackage{mathtools}
\usepackage{listings}
\usepackage{tikz}
\usetikzlibrary{shapes,arrows,positioning}
\usepackage{circuitikz}
\renewcommand{\vec}[1]{\boldsymbol{\mathbf{#1}}}
\DeclareMathOperator*{\Res}{Res}
\renewcommand\thesection{\arabic{section}}
\renewcommand\thesubsection{\thesection.\arabic{subsection}}
\renewcommand\thesubsubsection{\thesubsection.\arabic{subsubsection}}

\renewcommand\thesectiondis{\arabic{section}}
\renewcommand\thesubsectiondis{\thesectiondis.\arabic{subsection}}
\renewcommand\thesubsubsectiondis{\thesubsectiondis.\arabic{subsubsection}}
\hyphenation{op-tical net-works semi-conduc-tor}

\lstset{
language=Python,
frame=single, 
breaklines=true,
columns=fullflexible
}
\begin{document}
\theoremstyle{definition}
\newtheorem{theorem}{Theorem}[section]
\newtheorem{problem}{Problem}
\newtheorem{proposition}{Proposition}[section]
\newtheorem{lemma}{Lemma}[section]
\newtheorem{corollary}[theorem]{Corollary}
\newtheorem{example}{Example}[section]
\newtheorem{definition}{Definition}[section]
\newcommand{\BEQA}{\begin{eqnarray}}
        \newcommand{\EEQA}{\end{eqnarray}}
\newcommand{\define}{\stackrel{\triangle}{=}}
\newcommand{\myvec}[1]{\ensuremath{\begin{pmatrix}#1\end{pmatrix}}}
\newcommand{\mydet}[1]{\ensuremath{\begin{vmatrix}#1\end{vmatrix}}}
\bibliographystyle{IEEEtran}
\providecommand{\nCr}[2]{\,^{#1}C_{#2}} % nCr
\providecommand{\nPr}[2]{\,^{#1}P_{#2}} % nPr
\providecommand{\mbf}{\mathbf}
\providecommand{\pr}[1]{\ensuremath{\Pr\left(#1\right)}}
\providecommand{\qfunc}[1]{\ensuremath{Q\left(#1\right)}}
\providecommand{\sbrak}[1]{\ensuremath{{}\left[#1\right]}}
\providecommand{\lsbrak}[1]{\ensuremath{{}\left[#1\right.}}
\providecommand{\rsbrak}[1]{\ensuremath{{}\left.#1\right]}}
\providecommand{\brak}[1]{\ensuremath{\left(#1\right)}}
\providecommand{\lbrak}[1]{\ensuremath{\left(#1\right.}}
\providecommand{\rbrak}[1]{\ensuremath{\left.#1\right)}}
\providecommand{\cbrak}[1]{\ensuremath{\left\{#1\right\}}}
\providecommand{\lcbrak}[1]{\ensuremath{\left\{#1\right.}}
\providecommand{\rcbrak}[1]{\ensuremath{\left.#1\right\}}}
\theoremstyle{remark}
\newtheorem{rem}{Remark}
\newcommand{\sgn}{\mathop{\mathrm{sgn}}}
\newcommand{\rect}{\mathop{\mathrm{rect}}}
\newcommand{\sinc}{\mathop{\mathrm{sinc}}}
\providecommand{\abs}[1]{\left\vert#1\right\vert}
\providecommand{\res}[1]{\Res\displaylimits_{#1}}
\providecommand{\norm}[1]{\lVert#1\rVert}
\providecommand{\mtx}[1]{\mathbf{#1}}
\providecommand{\mean}[1]{E\left[ #1 \right]}
\providecommand{\fourier}{\overset{\mathcal{F}}{ \rightleftharpoons}}
\providecommand{\ztrans}{\overset{\mathcal{Z}}{ \rightleftharpoons}}
\providecommand{\system}[1]{\overset{\mathcal{#1}}{ \longleftrightarrow}}
\newcommand{\solution}{\noindent \textbf{Solution: }}
\providecommand{\dec}[2]{\ensuremath{\overset{#1}{\underset{#2}{\gtrless}}}}
\let\StandardTheFigure\thefigure
\def\putbox#1#2#3{\makebox[0in][l]{\makebox[#1][l]{}\raisebox{\baselineskip}[0in][0in]{\raisebox{#2}[0in][0in]{#3}}}}
\def\rightbox#1{\makebox[0in][r]{#1}}
\def\centbox#1{\makebox[0in]{#1}}
\def\topbox#1{\raisebox{-\baselineskip}[0in][0in]{#1}}
\def\midbox#1{\raisebox{-0.5\baselineskip}[0in][0in]{#1}}

\vspace{3cm}
\title{12.11.3.9}
\author{Lokesh Surana}
\maketitle
\section*{Class 12, Chapter 11, Exercise 4.19}


\solution
The line equations are given as
\begin{align}
    \label{eq:chapters/12/11/4/19/1} \vec{r} = \vec{A} + \lambda\vec{m}
\end{align}
and $\vec{A}$ is any point on the line. 
\fi
The normal vectors of the planes are
\begin{align}
\vec{n}_1 = \myvec{1\\-1\\2},\,
    \vec{n}_2 = \myvec{3\\1\\1}.
\end{align}
Since the desired line 
is parallel to both the planes,
\begin{align}
    \label{eq:chapters/12/11/4/19/6} \vec{n}_1^{\top}\vec{m} = 0 \\
    \label{eq:chapters/12/11/4/19/7} \vec{n}_2^{\top}\vec{m} = 0 \\
    \label{eq:chapters/12/11/4/19/8} \implies \myvec{1&-1&2 \\ 3&1&1}\vec{m} = 0
\end{align}
where $\vec{m}$ is the direction vector of the line. 
Let's reduce the matrix from equation \eqref{eq:chapters/12/11/4/19/8} to row-echelon form
\begin{align}
    \label{eq:chapters/12/11/4/19/9} \myvec{1&-1&2 \\ 3&1&1} &\xleftrightarrow[]{R_2\rightarrow -\frac{3}{4}{R_1} + \frac{1}{4}{R_2}} \myvec{1&-1&2 \\ 0&1&-\frac{5}{4}}\\
    \label{eq:chapters/12/11/4/19/10} \myvec{1&-1&2 \\ 0&1&-\frac{5}{4}} &\xleftrightarrow[]{R_1\rightarrow {R_1} + {R_2}} \myvec{1&0&\frac{3}{4} \\ 0&1&-\frac{5}{4}}
\end{align}
Using \eqref{eq:chapters/12/11/4/19/8}, \eqref{eq:chapters/12/11/4/19/9} and \eqref{eq:chapters/12/11/4/19/10}, 
\begin{align}
    \myvec{1&0&\frac{3}{4} \\ 0&1&-\frac{5}{4}}\vec{m} &= 0 \\
    \implies \myvec{{m}_1\\{m}_2\\{m}_3} &= \myvec{-\frac{3}{4}{m}_3\\\frac{5}{4}{m}_3\\{m}_3} \\
    \implies \myvec{{m}_1\\{m}_2\\{m}_3} &= {m}_3\myvec{-\frac{3}{4}\\\frac{5}{4}\\1} \\
    \implies \vec{m} = \myvec{-3\\5\\4}
\end{align}
It is given that the line passes through the point $\myvec{1\\2\\3}$, so the final equation of line is
\begin{align}
    \vec{x} = \myvec{1\\2\\3} + \lambda\myvec{-3\\5\\4} 
\end{align}

