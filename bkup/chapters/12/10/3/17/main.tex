\iffalse
\documentclass[journal,12pt,twocolumn]{IEEEtran}

\usepackage{setspace}
\usepackage{gensymb}
\usepackage{xcolor}
\usepackage{caption}
\singlespacing
\usepackage{siunitx}
\usepackage[cmex10]{amsmath}
\usepackage{mathtools}
\usepackage{hyperref}
\usepackage{amsthm}
\usepackage{mathrsfs}
\usepackage{txfonts}
\usepackage{stfloats}
\usepackage{cite}
\usepackage{cases}
\usepackage{subfig}
\usepackage{longtable}
\usepackage{multirow}
\usepackage{enumitem}
\usepackage{mathtools}
\usepackage{listings}
\usepackage{tasks}
\usepackage{siunitx}
\sisetup{
   detect-family,
   detect-inline-family=math,
}
\usepackage{tikz}
\usetikzlibrary{shapes,arrows,positioning}
\usepackage{circuitikz}
\let\vec\mathbf
\DeclareMathOperator*{\Res}{Res}
\renewcommand\thesection{\arabic{section}}
\renewcommand\thesubsection{\thesection.\arabic{subsection}}
\renewcommand\thesubsubsection{\thesubsection.\arabic{subsubsection}}

\renewcommand\thesectiondis{\arabic{section}}
\renewcommand\thesubsectiondis{\thesectiondis.\arabic{subsection}}
\renewcommand\thesubsubsectiondis{\thesubsectiondis.\arabic{subsubsection}}
\hyphenation{op-tical net-works semi-conduc-tor}

\lstset{
language=Python,
frame=single, 
breaklines=true,
columns=fullflexible
}
\begin{document}
\theoremstyle{definition}
\newtheorem{theorem}{Theorem}[section]
\newtheorem{problem}{Problem}
\newtheorem{proposition}{Proposition}[section]
\newtheorem{lemma}{Lemma}[section]
\newtheorem{corollary}[theorem]{Corollary}
\newtheorem{example}{Example}[section]
\newtheorem{definition}{Definition}[section]
\newcommand{\BEQA}{\begin{eqnarray}}
\newcommand{\EEQA}{\end{eqnarray}}
\newcommand{\define}{\stackrel{\triangle}{=}}
\newcommand{\myvec}[1]{\ensuremath{\begin{pmatrix}#1\end{pmatrix}}}
\newcommand{\mydet}[1]{\ensuremath{\begin{vmatrix}#1\end{vmatrix}}}

\bibliographystyle{IEEEtran}
\providecommand{\nCr}[2]{\,^{#1}C_{#2}} % nCr
\providecommand{\nPr}[2]{\,^{#1}P_{#2}} % nPr
\providecommand{\mbf}{\mathbf}
\providecommand{\pr}[1]{\ensuremath{\Pr\left(#1\right)}}
\providecommand{\qfunc}[1]{\ensuremath{Q\left(#1\right)}}
\providecommand{\sbrak}[1]{\ensuremath{{}\left[#1\right]}}
\providecommand{\lsbrak}[1]{\ensuremath{{}\left[#1\right.}}
\providecommand{\rsbrak}[1]{\ensuremath{{}\left.#1\right]}}
\providecommand{\brak}[1]{\ensuremath{\left(#1\right)}}
\providecommand{\lbrak}[1]{\ensuremath{\left(#1\right.}}
\providecommand{\rbrak}[1]{\ensuremath{\left.#1\right)}}
\providecommand{\cbrak}[1]{\ensuremath{\left\{#1\right\}}}
\providecommand{\lcbrak}[1]{\ensuremath{\left\{#1\right.}}
\providecommand{\rcbrak}[1]{\ensuremath{\left.#1\right\}}}
\theoremstyle{remark}
\newtheorem{rem}{Remark}
\newcommand{\sgn}{\mathop{\mathrm{sgn}}}
\newcommand{\rect}{\mathop{\mathrm{rect}}}
\newcommand{\sinc}{\mathop{\mathrm{sinc}}}
\providecommand{\abs}[1]{\left\vert#1\right\vert}
\providecommand{\res}[1]{\Res\displaylimits_{#1}} 
\providecommand{\norm}[1]{\lVert#1\rVert}
\providecommand{\mtx}[1]{\mathbf{#1}}
\providecommand{\mean}[1]{E\left[ #1 \right]}
\providecommand{\fourier}{\overset{\mathcal{F}}{ \rightleftharpoons}}
\providecommand{\ztrans}{\overset{\mathcal{Z}}{ \rightleftharpoons}}
\providecommand{\system}[1]{\overset{\mathcal{#1}}{ \longleftrightarrow}}
\newcommand{\solution}{\noindent \textbf{Solution: }}
\providecommand{\dec}[2]{\ensuremath{\overset{#1}{\underset{#2}{\gtrless}}}}
\let\StandardTheFigure\thefigure
\def\putbox#1#2#3{\makebox[0in][l]{\makebox[#1][l]{}\raisebox{\baselineskip}[0in][0in]{\raisebox{#2}[0in][0in]{#3}}}}
     \def\rightbox#1{\makebox[0in][r]{#1}}
     \def\centbox#1{\makebox[0in]{#1}}
     \def\topbox#1{\raisebox{-\baselineskip}[0in][0in]{#1}}
     \def\midbox#1{\raisebox{-0.5\baselineskip}[0in][0in]{#1}}

\vspace{3cm}
\title{12.10.3.17}
\author{Lokesh Surana}
\maketitle
\section*{Class 12, Chapter 10, Exercise 3.17}
\begin{enumerate}[start=17]
\item  Show that the vectors form the vertices $\myvec{2\\-1\\1}$, $\myvec{1\\-3\\-5}$ and $\myvec{3\\-4\\-4}$ of a right angled triangle.
\break
\solution 
\fi
		Let's first check if points are not collinear and hence forming a triangle, using row reduction.

\begin{align}
 \myvec{2&1&3\\-1&-3&-4\\1&-5&-4\\1&1&1} &\xleftrightarrow[]{R_2\rightarrow \frac{R_1}{2}+R_2} \myvec{2&1&3\\0&-\frac{5}{2}&-\frac{5}{2}\\1&-5&-4\\1&1&1}\\
 &\xleftarrow[]{R_3\rightarrow -\frac{R_1}{2} + R_3} \myvec{2&1&3\\0&-\frac{5}{2}&-\frac{5}{2}\\0&-\frac{11}{2}&-\frac{11}{2}\\1&1&1} \\
 &\xleftarrow[]{R_4\rightarrow -\frac{R_1}{2} + R_4} \myvec{2&1&3\\0&-\frac{5}{2}&-\frac{5}{2}\\0&-\frac{11}{2}&-\frac{11}{2}\\0&\frac{1}{2}&-\frac{1}{2}} \\
 &\xleftarrow[]{R_3\rightarrow -\frac{11R_2}{5} + R_3} \myvec{2&1&3\\0&-\frac{5}{2}&-\frac{5}{2}\\0&0&0\\0&\frac{1}{2}&-\frac{1}{2}} \\
 &\xleftarrow[]{R_4\rightarrow \frac{R_2}{5} + R_4} \myvec{2&1&3\\0&-\frac{5}{2}&-\frac{5}{2}\\0&0&0\\0&0&-1} 
\end{align}
Hence Rank of matrix is 3, which means that points are not collinear. Now, let's check if they form a right angled triangle. 
Let 
\begin{align}
\vec{A} = \myvec{2\\-1\\1},\, \vec{B} = \myvec{1\\-3\\-5},\,\vec{C} = \myvec{3\\-4\\-4}
\end{align}
Then,
\begin{align}
 \vec{a} &= \vec{C} - \vec{B} = \myvec{2\\-1\\1}\\
 \vec{b} &= \vec{A} - \vec{C} = \myvec{-1\\3\\5}\\
 \vec{c} &= \vec{B} - \vec{A} = \myvec{-1\\-2\\-6}
\end{align}

If inner product of two vectors is zero, then they are perpendicular. So, we have:

\begin{align}
    \vec{a}^\top \vec{b} &= 2\times(-1) + (-1)\times3 + 1\times5 = 0\\
    \vec{b}^\top \vec{c} &= (-1)\times(-1) + 3\times(-2) + 5\times(-6) = -35\\
    \vec{c}^\top \vec{a} &= (-1)\times2 + (-2)\times(-1) + (-6)\times1 = -6
\end{align}

So, $\vec{a}$ and $\vec{b}$ are perpendicular and therefore triangle $ABC$ is right angled triangle.

