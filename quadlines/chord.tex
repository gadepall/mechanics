%\begin{enumerate}
\begin{enumerate}[label=\thesection.\arabic*.,ref=\thesection.\theenumi]
		\item
  The points of intersection of the line 
\begin{align}
L: \quad \vec{x} = \vec{h} + \mu \vec{m} \quad \mu \in \mathbb{R}
\label{eq:conic_tangent}
\end{align}
with the conic section in \eqref{eq:conic_quad_form} are given by
\begin{align}
\vec{x}_i = \vec{h} + \mu_i \vec{m}
	\label{eq:chord-pts}
\end{align}
%
where
\begin{multline}
\mu_i = \frac{1}
{
\vec{m}^{\top}\vec{V}\vec{m}
}
\lbrak{-\vec{m}^{\top}\brak{\vec{V}\vec{h}+\vec{u}}}
\\
\pm
{\small
\rbrak{\sqrt{
\sbrak{
\vec{m}^{\top}\brak{\vec{V}\vec{h}+\vec{u}}
}^2
	-\text{g}
\brak
{\vec{h}
%\vec{h}^{\top}\vec{V}\vec{h} + 2\vec{u}^{\top}\vec{h} +f
}
\brak{\vec{m}^{\top}\vec{V}\vec{m}}
}
}
}
\label{eq:tangent_roots}
\end{multline}



\begin{proof}
  Substituting \eqref{eq:conic_tangent}
in \eqref{eq:conic_quad_form}, 
\begin{align}
\brak{\vec{h} + \mu \vec{m}}^{\top}\vec{V}\brak{\vec{h} + \mu \vec{m}}  + 2 \vec{u}^{\top}\brak{\vec{h} + \mu \vec{m}}+f &= 0
\\
\implies \mu^2\vec{m}^{\top}\vec{V}\vec{m} + 2 \mu\vec{m}^{\top}\brak{\vec{V}\vec{h}+\vec{u}} 
+ \vec{h}^{\top}\vec{V}\vec{h} + 2\vec{u}^{\top}\vec{h} +f &= 0
	\\
	\text{or, }
\mu^2\vec{m}^{\top}\vec{V}\vec{m} + 2 \mu\vec{m}^{\top}\brak{\vec{V}\vec{h}+\vec{u}} 
	+ \text{g}\brak{\vec{h}} &=0
	%^{\top}\vec{V}\vec{h} + 2\vec{u}^{\top}\vec{h} +f &= 0
\label{eq:conic_intercept}
\end{align}
for g defined in \eqref{eq:conic_quad_form}.
Solving the above quadratic in \eqref{eq:conic_intercept}
yields \eqref{eq:tangent_roots}.
\end{proof}
\item
  If $L$ in \eqref{eq:conic_tangent} touches \eqref{eq:conic_quad_form} at exactly one point $\vec{q}$, 
  \begin{align}
  \vec{m}^{\top}\brak{\vec{V}\vec{q}+\vec{u}} = 0
  \label{eq:conic_tangent_mq}
  \end{align}

\begin{proof}
  In this case, \eqref{eq:conic_intercept} has exactly one root.  Hence, 
  in \eqref{eq:tangent_roots}
  \begin{align}
  \sbrak{
  \vec{m}^{\top}\brak{\vec{V}\vec{q}+\vec{u}}
  }^2 -\brak{\vec{m}^{\top}\vec{V}\vec{m}}
	  \text{g}\brak
  {
  \vec{q}
%  \vec{q}^{\top}\vec{V}\vec{q} + 2\vec{u}^{\top}\vec{q} +f
  } = 0                                                                                             
  \label{eq:conic_tangent_disc}
  \end{align}                    
  $\because \vec{q}$ is the point of contact,
	%$\vec{q}$ satisfies \eqref{eq:conic_quad_form}
%  and 
  \begin{align}
	  \text{g}\brak{  \vec{q}} = 0
%  \vec{q}^{\top}\vec{V}\vec{q} + 2\vec{u}^{\top}\vec{q} +f = 0
  \label{eq:conic_tangent_qquad}
  \end{align}
  Substituting \eqref{eq:conic_tangent_qquad} in \eqref{eq:conic_tangent_disc} and simplifying, we obtain \eqref{eq:conic_tangent_mq}.
\end{proof}
	\item
		The length of the chord in 
\eqref{eq:conic_tangent}
is given by 
\begin{align}
 \frac{2\sqrt{
\sbrak{
\vec{m}^{\top}\brak{\vec{V}\vec{h}+\vec{u}}
}^2
-
\brak
{
\vec{h}^{\top}\vec{V}\vec{h} + 2\vec{u}^{\top}\vec{h} +f
}
\brak{\vec{m}^{\top}\vec{V}\vec{m}}
}
}
{
\vec{m}^{\top}\vec{V}\vec{m}
}\norm{\vec{m}}
\label{eq:chord-len}
  \end{align}
	
\begin{proof}
The distance between the points in 
	\eqref{eq:chord-pts}
is given by 
\begin{align}
	\norm{\vec{x}_1-\vec{x}_2} =  \abs{\mu_1-\mu_2} \norm{\vec{m}}
\label{eq:conic_tangent_pts_dist}
\end{align}
Substituing $\mu_i$ from 
\eqref{eq:tangent_roots} in
\eqref{eq:conic_tangent_pts_dist}
yields
	\eqref{eq:chord-len}.
\end{proof}
	\item
 The affine transform for the conic section, preserves the norm.  This implies that the length of any chord of a conic
	is invariant to translation and/or rotation.
	
	\begin{proof}
	Let 
%From \eqref{eq:conic_affine}, 
\begin{align}
\vec{x}_i = \vec{P}\vec{y}_i+\vec{c} 
\label{eq:conic_affine_pts}
\end{align}
be any two points on the conic.  Then the distance between the points is given by 
\begin{align}
	\norm{\vec{x}_1-\vec{x}_2 } &= \norm{\vec{P}\brak{	\vec{y}_1 -\vec{y}_2 }}
\end{align}
which can be expressed as 
\begin{align}
	\norm{\vec{x}_1-\vec{x}_2 }^2 &= 		\brak{\vec{y}_1 -\vec{y}_2 }^{\top}\vec{P}^{\top}\vec{P}\brak{\vec{y}_1 -\vec{y}_2 }
	\\
	&= 		\norm{\vec{y}_1 -\vec{y}_2 }^2
\label{eq:conic_affine_norm_preserve}
\end{align}
since 
\begin{align}
	\vec{P}^{\top}\vec{P} = \vec{I}
\end{align}
	\end{proof}
    \item For the standard hyperbola/ellipse, the length of the major axis is 
  \begin{align}
\label{eq:chord-len-major}
 2\sqrt{\abs{\frac{
f_0}
{\lambda_1}
	  }}
  \end{align}
  and the minor axis is 
  \begin{align}
\label{eq:chord-len-minor}
 2\sqrt{\abs{\frac{
f_0}
{\lambda_2}
	  }}
  \end{align}
%	    $\mydet{\vec{V}} \ne 0$, the lengths of the semi-major and semi-minor axes of the conic in \eqref{eq:conic_quad_form} are given by 
%  \begin{align} 
%    \label{eq:ellipse_axes}
%  %  \begin{aligned}[t]
%    \sqrt{\frac{\vec{u}^{\top}\vec{V}^{-1}\vec{u} -f}{\lambda_1}}, 
%    \sqrt{\frac{\vec{u}^{\top}\vec{V}^{-1}\vec{u} -f}{\lambda_2}}. \quad \brak{\text{ellipse}}
%    \\
%%
%       \sqrt{\frac{\vec{u}^{\top}\vec{V}^{-1}\vec{u} -f}{\lambda_1}}, 
%       \sqrt{\frac{f-\vec{u}^{\top}\vec{V}^{-1}\vec{u}}{\lambda_2}}, \quad \brak{\text{hyperbola }}
%%
%  %\end{aligned}
%  \label{eq:hyper_axes}
%\end{align} 
%\solution For \begin{align} \abs{\vec{V}} > 0, \quad \text{or, } \lambda_1 > 0, \lambda_2 > 0 
%  \end{align} and \eqref{eq:conic_simp_temp_nonparab} becomes 
%  \begin{align} 
%	  \lambda_1y_1^2 +\lambda_2y_2^2 = 
%  \vec{u}^{\top}\vec{V}^{-1}\vec{u} -f 
%	  \label{eq:hyper-pair-cond}
%  \end{align} 
%  yielding        \eqref{eq:ellipse_axes}.  Similarly, \eqref{eq:hyper_axes} can be obtained for 
%  \begin{align} 
%    \label{eq:conic_hyper_cond}
%    \abs{\vec{V}} 
%    < 0, \quad \text{or, } \lambda_1 > 0, \lambda_2 < 0 \end{align}

\begin{proof}
%	See Appendix \ref{app:major}
		\label{app:major}
		Since the major axis passes through the origin, 
  \begin{align}
	  \vec{q} =			\vec{0} 
\end{align}  
Further, from Corollary  
		\eqref{corr:axis},
  \begin{align}
  \vec{m}&= \vec{e}_2,  
\end{align} and
from 
    \eqref{eq:conic_simp_temp_nonparab},
  \begin{align}
	  \vec{V} =     \frac{\vec{D} }{f_0}, 
	   \vec{u} = 0, 
	   f = -1
	    \label{eq:latus_rectum_ellipse_param}
\end{align}  
Substituting the above in
\eqref{eq:chord-len}, 
\begin{align}
 \frac{2\sqrt{
\vec{e}_1^{\top}\frac{\vec{D}}{f_0}\vec{e}_1
}
}
{
\vec{e}_1^{\top}\frac{\vec{D}}{f_0}\vec{e}_1
}\norm{\vec{e}_1}
  \end{align}
  yielding 
\eqref{eq:chord-len-major}.
Similarly, for the minor axis, the only different parameter is 
  \begin{align}
  \vec{m}&= \vec{e}_2,  
\end{align} 
Substituting the above in
\eqref{eq:chord-len}, 
\begin{align}
 \frac{2\sqrt{
\vec{e}_2^{\top}\frac{\vec{D}}{f_0}\vec{e}_2
}
}
{
\vec{e}_2^{\top}\frac{\vec{D}}{f_0}\vec{e}_2
}\norm{\vec{e}_2}
  \end{align}
  yielding 
\eqref{eq:chord-len-minor}.

\end{proof}
\item
    The latus rectum of a conic section is the chord that passes through the focus and is perpendicular to the major axis.
	The length of the latus rectum for a conic is given by
		\begin{align}
			l =
			\begin{cases}
				2\frac{\sqrt{\abs{f_0\lambda_1}}}{\lambda_2} & e \ne 1
			\\
			\frac{\eta}{\lambda_2} & e = 1
			\end{cases}
			\label{eq:latus-ellipse}
		\end{align}

		\begin{proof}
%			See Appendix \ref{app:latus}.
		%\section{}
		\label{app:latus}
			The latus rectum is perpendicular to the major axis for the standard conic.  Hence, from Corollary  
		\eqref{corr:axis},
  \begin{align}
  \vec{m}&= \vec{e}_2,  
\end{align}  
Since it passes through the focus, from 
					\eqref{eq:F-ell-hyp-parab}
  \begin{align}
	  \vec{q} =			\vec{F} 
=
					 \pm e\sqrt{\frac{f_0}{\lambda_2\brak{1-e^2}}} \vec{e }_1
%					 \frac{e}{\sqrt{f_0\lambda_2\brak{1-e^2}}}\vec{e }_1
\end{align}  
for the standard hyperbola/ellipse.  Also, 
from 
    \eqref{eq:conic_simp_temp_nonparab},
  \begin{align}
	  \vec{V} =     \frac{\vec{D} }{f_0}, 
	   \vec{u} = 0, 
	   f = -1
	    \label{eq:latus_rectum_ellipse_param-new}
\end{align}  
Substituting the above in
\eqref{eq:chord-len}, 
\begin{align}
 \frac{2\sqrt{
\sbrak{
\vec{e}_2^{\top}\brak{\frac{\vec{D}}{f_0} e\sqrt{\frac{f_0}{\lambda_2\brak{1-e^2}}} \vec{e }_1}
}^2
-
\brak
{
 e\sqrt{\frac{f_0}{\lambda_2\brak{1-e^2}}} \vec{e }_1^{\top}\frac{\vec{D}}{f_0} e\sqrt{\frac{f_0}{\lambda_2\brak{1-e^2}}} \vec{e }_1 -1 
}
\brak{\vec{e}_2^{\top}\frac{\vec{D}}{f_0}\vec{e}_2}
}
}
{
\vec{e}_2^{\top}\frac{\vec{D}}{f_0}\vec{e}_2
}\norm{\vec{e}_2}
\label{eq:chord-len-sub-ell}
  \end{align}
  Since 
  \begin{align}
\vec{e}_2^{\top}\vec{D}\vec{e}_1 = 0, 
%\vec{e}_2^{\top}\vec{e}_2 = 0,
\vec{e}_1^{\top}\vec{D}\vec{e}_1 = \lambda_1,
\vec{e}_1^{\top}\vec{e}_1 = 1,
	  \norm{\vec{e}_2} = 1,
\vec{e}_2^{\top}\vec{D}\vec{e}_2 = \lambda_2,
  \end{align}
\eqref{eq:chord-len-sub-ell} can be expressed as 
  \begin{align}
	&		\frac{2\sqrt{\brak{1-\frac{\lambda_1e^2}{{\lambda_2\brak{1-e^2}}}}\brak{\frac{\lambda_2}{f_0}}}}
{
	\frac{\lambda_2}{f_0}
	} 	
	\\
	&=		2\frac{\sqrt{
		f_0\lambda_1}}{\lambda_2}
 & \brak{ \because e^2 = 1-\frac{\lambda_1}{\lambda_2}}
		   \end{align}
For the standard parabola, the parameters in 
\eqref{eq:chord-len} are
\begin{align}  
	\vec{q} =\vec{F} =  -\frac{\eta}{4\lambda_2}\vec{e}_1, \vec{m} = \vec{e}_1, \vec{V} = \vec{D},
	\vec{u} = \frac{\eta}{2}\vec{e}_1^{\top}, f = 0
\end{align}  

Substituting the above in
\eqref{eq:chord-len}, 
%			from \eqref{eq:conic_simp_temp_nonparab},  
%					from \eqref{eq:F-ell-hyp-parab}
%and 						 \\
the length of the latus rectum  can be expressed as
{\footnotesize
\begin{align}
 \frac{2\sqrt{
\sbrak{
\vec{e}_2^{\top}\brak{\vec{D}\brak{-\frac{\eta}{4\lambda_2}\vec{e}_1}+\frac{\eta}{2}\vec{e}_1}
}^2
-
\brak
{
\brak{-\frac{\eta}{4\lambda_2}\vec{e}_1}^{\top}\vec{D}\brak{-\frac{\eta}{4\lambda_2}\vec{e}_1} + 2\frac{\eta}{2}\vec{e}_1^{\top}\brak{-\frac{\eta}{4\lambda_2}\vec{e}_1} 
}
\brak{\vec{e}_2^{\top}\vec{D}\vec{e}_2}
}
}
{
\vec{e}_2^{\top}\vec{D}\vec{e}_2
}\norm{\vec{e}_2}
\label{eq:chord-len-sub}
  \end{align}
  }
  Since 
  \begin{align}
\vec{e}_2^{\top}\vec{D}\vec{e}_1 = 0, 
\vec{e}_2^{\top}\vec{e}_2 = 0,
\vec{e}_1^{\top}\vec{D}\vec{e}_1 = 0,
\vec{e}_1^{\top}\vec{e}_1 = 1,
	  \norm{\vec{e}_1} = 1,
\vec{e}_2^{\top}\vec{D}\vec{e}_2 = \lambda_2,
  \end{align}
\eqref{eq:chord-len-sub} can be expressed as 
  \begin{align}
	  2 \frac{\sqrt{\frac{\eta^2}{4\lambda_2}\lambda_2}}{\lambda_2}
	  = \frac{\eta}{\lambda_2}
  \end{align}
\end{proof}
\end{enumerate}
