\begin{enumerate}[label=\thesection.\arabic*,ref=\thesection.\theenumi]
\numberwithin{equation}{enumi}
\numberwithin{figure}{enumi}
\numberwithin{table}{enumi}
\item The perpendicular bisector of the line segment joining the points $\vec{A} (1, 5) \text{ and }
\vec{B} (4, 6)$ cuts the y-axis at
\begin{enumerate}
	\item$(0, 13)$ 
	\item $(0, –13)$
	\item$(0, 12) $
	\item$(13, 0)$
\end{enumerate}
\item The point which lies on the perpendicular bisector of the line segment joining the
	points $\vec{A} (–2, –5)\text { and } \vec{B} (2, 5) $is
\begin{enumerate}
\item  	$(0, 0)$
\item  $(0, 2)$ 
\item  $(2, 0)$ 
\item  $(–2, 0)$
\end{enumerate}
\item The points $ (–4, 0), (4, 0), (0, 3) $are the vertices of
	\begin{enumerate}
\item right triangle 
\item isosceles triangle
\item  equilateral triangle
\item  scalene triangle 
\end{enumerate}
\item The point $\vec{A}(2,7)$ lies on the perpendicular bisector of line segment joining the points $\vec{P}(6,5)\text{ and } \vec{Q}(0,-4)$
\item The points $\vec{A}(-1,-2),\vec{B}(4,3),\vec{C}(2,5) \text{ and } \vec{D}(-3,0)$ in that order a rectangle
\item Name the type of triangle formed by the points $\vec{A}(-5,6),\vec{B}(-4,-2),\text{ and }\vec{C}(7,5)$.
\item What type of a quadrilateral do the points $\vec{A}(2,-2),\vec{B}(7,3),\vec{C}(11,-1),\text{ and }\vec{D}(6,-6)$ taken in that order,form?
\item Find the coordinates of the point $\vec{Q}$ on the $x$-axis which lies on the perpendicular bisector of the line segment joining the points $\vec{A}(-5,-2) \text{ and }{B}(4,-2)$.Name the type of triangle formed by points $\vec{Q},\vec{A}\text{ and }\vec{B}$.
\item The points $\vec{A}(2,9),\vec{B}(a,5) \text{ and }\vec{C}(5,5)$ are the verices of a triangle $\vec{ABC}$ right angled at $\vec{B}$. Find the values of a and hence the area of $\triangle \vec{ABC}$.
\item Find a vector of magnitude 6, which is perpendicular to both the vectors $2\hat{i}-\hat{j}$+$2\hat{k}$ $\text{ and }$ $4\hat{i}-\hat{j}+3\hat{k}$.
\item If A,B,C,D  are the points with position vectors $\hat{i}+\hat{j}-\hat{k}$, $2\hat{i}-\hat{j}+3\hat{k}$, $2\hat{i}-3\hat{k}$, $3\hat{i}$-$2\hat{j}$+$\hat{k}$, respectively, find the projection of $\overline{AB}$ $\text{ along }$ $\overline{CD}$.
\item Find the value of $\lambda$ such that the vectors $\vec{a}=2\hat{i}+\lambda\hat{j}+\hat{k}$ $\text{and}$ $\vec{b}=\hat{i}+2\hat{j}+3\hat{k}$ are orthogonal.
	\begin{enumerate}
\item 0
\item 1 
\item $\frac{3}{2}$
\item $-\frac{5}{2}$
	\end{enumerate}
\item Projection vector of $\vec{a}$ on $\vec{b}$ is
	\begin{enumerate}
\item $\left(\dfrac{\vec{a}.\vec{b}}{\abs{\vec{b}}^2}\right)$
\item $\frac{\vec{a}.\vec{b}}{\abs{\vec{b}}}$
\item $\frac{\vec{a}.\vec{b}}{\abs{\vec{a}}}$
\item $\left(\dfrac{\vec{a}.\vec{b}}{\abs{\vec{a}}^2}\right)$
\end{enumerate}
\item The vectors $\lambda\hat{i}+\lambda\hat{j}+2\hat{k}$, $\hat{i}+\lambda\hat{j}-\hat{k}$ $\text{ and }$ $2\hat{i}-\hat{j}+\lambda\hat{k}$ are coplanar if
	\begin{enumerate}
\item	$\lambda=-2$
\item $\lambda=0$
\item $\lambda=1$
\item	$\lambda=-1$
\end{enumerate}
\item The number of vectors of unit length perpendicular to the vectors $\vec{a}=2\hat{i}+\hat{j}+2\hat{k}$ $\text{ and }$ $\vec{b}=\hat{j}+\hat{k}$ is
	\begin{enumerate}
\item one
\item  two
\item three
\item infinite
\end{enumerate}
\item If $\vec{r}.\vec{a}=0$, $\vec{r}.\vec{b}=0$ $\text{and}$ $\vec{r}.\vec{c}=0$ for some non-zero vector $\vec{r}$, then the value of $\vec{a}.(\vec{b}$ $\times$ $\vec{c})$ is \rule{1cm}{0.15mm}.
\item If $\abs{\vec{a}+\vec{b}}$ = $\abs{\vec{a}-\vec{b}}$, then the vectors $\vec{a}$ $\text {and}$ $\vec{b}$ are orthogonal.
\item Prove that the lines $x=py+q , z=ry+s \text{ and } x=p^{\prime}y+q^{\prime}, z=r^{\prime}y+s^{\prime} $ are perpendicular if $pp^{\prime}+rr^{\prime}+1=0$.
\item Find the equation of a plane which  bisects perpendicularly the line joining the points A$(2,3,4)$ and B$(4,5,8)$ at right angles.
\item $\overrightarrow{AB}=3\hat{i}-\hat{j}+\hat{k}$ and $\overrightarrow{CD}=-3\hat{i}+2\hat{j}+4\hat{k}$ are two vectors. The position vectors of the points A and C are $6\hat{i}+7\hat{j}+4\hat{k}$ and $-9\hat{j}+2\hat{k},$ respectively. Find the position vector of a point P on the line AB and a point Q on the line CD such that $\overrightarrow{PQ}$ is perpendicular to $\overrightarrow{AB}$ and $\overrightarrow{CD}$ both.
\item Show that the straight lines whose direction cosines are given by $2l+2m-n=0$ and $mn+nl+lm=0$ are at right angles.
\item If $l_1, m_1, n_1;l_2, m_2, n_2;l_3, m_3, n_3$ are the direction cosines of the three mutually perpendcular lines, prove that the line whose direction cosines are propotional to $l_1+l_2+l_3 , m_1+m_2,m_3, n_1+n_2+n_3$ make angles with them.
\item The intercepts made by the plane $2x-3y+5z+4=0$ on the co-ordinate axis are $\brak{-2,\dfrac{4}{3},-\dfrac{4}{5}}$.
\item The line $\overrightarrow{r}=2\hat{i}-3\hat{j}-\hat{k}+\lambda(\hat{i}-\hat{j}+2\hat{k})$ lies in the plane $\overrightarrow{r} \cdot (3\hat{i}+\hat{j}-\hat{k})+2=0$.
\end{enumerate}
