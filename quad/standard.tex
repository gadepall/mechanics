%\begin{enumerate}
\begin{enumerate}[label=\thesection.\arabic*.,ref=\thesection.\theenumi]
	\item
			\label{corr:center}
			The center of the standard ellipse/hyperbola, defined to be the mid point of the line joining the foci, is the origin.
	
	\item
		\label{corr:axis}
			The principal (major) axis of the standard ellipse/hyperbola, defined to be the line joining the two foci   is the $x$-axis.  
	\begin{proof}
		From 	\eqref{eq:F-ell-hyp-parab}, it is obvious that the line joining the foci passes through the origin.  Also, the direction vector of this line is $\vec{e}_1$.  Thus, the principal axis is the $x$-axis. 
	\end{proof}
	\item
		\label{corr:minor-axis}
			The minor axis of the standard ellipse/hyperbola, defined to be the line orthogonal to the $x$-axis is the $y$-axis. 
\item The vertices of the standard ellipse/hyperbola, defined to be the points of intersection of the major axis with the conic are
\begin{align}
 \pm \frac{\vec{e}_1}{\sqrt{\lambda}_1}
	\label{eq:std-nonparab-vert}
\end{align}
		\solution Since the 
major axis can be expressed as
\begin{align}
	\vec{x} = {\mu} \vec{e}_1,
\end{align}
substituting
\begin{align}
\vec{h} = \vec{0}, \vec{m} = \vec{e}_1
\end{align}
in
\eqref{eq:conic_intercept},
yields
\begin{align}
	\mu^2\lambda_1  - 1 = 0 \implies \mu = \pm \frac{1}{\sqrt{\lambda}_1}
\end{align}
%
resulting in 
	\eqref{eq:std-nonparab-vert}.
\item The length of the major axis for the standard ellipse/hyperbola is 
\begin{align}
	\label{eq:std-nonparab-major-len}.
	\frac{2}{\sqrt{\lambda}_1}
\end{align}
\item The length of the minor axis for the standard ellipse/hyperbola is 
\begin{align}
	\label{eq:std-nonparab-minor-len}.
	\frac{2}{\sqrt{\lambda}_2}
\end{align}
	\item
			The axis of symmetry of the standard parabola, defined to be the line perpendicular to the directrix and passing through the focus,  is the $x$- axis.
	
	\begin{proof}
	From \eqref{eq:n-parab} and 	
					\eqref{eq:F-ell-hyp-parab}, 
					the axis of the parabola  can be expressed using 
    \eqref{eq:line_norm_eq} as 
		\begin{align}
			\vec{e}_2^{\top}\brak{\vec{y}  
			+\frac{\eta}{4\lambda_2}\vec{e}_1} &= 0
			\\
			\implies \vec{e}_2^{\top}\vec{y} &= 0
					\label{eq:axis-std-parab}, 
		\end{align}
		which is the equation of the $x$-axis.
	\end{proof}


	\item
			\label{corr:center-parab}
 The point where the parabola intersects its axis of symmetry is called the vertex. For the standard parabola, the vertex is the origin.
	
	\begin{proof}
					\eqref{eq:axis-std-parab} can be expressed as 
    \begin{align}
			\vec{y}= \alpha \vec{e}_1 
					\label{eq:axis-std-parab-dir}, 
    \end{align}
    using 
    \eqref{eq:line_norm_eq}.
					Substituting \eqref{eq:axis-std-parab-dir} in 
    \eqref{eq:conic_simp_temp_parab}, 
    \begin{align}
	     \alpha^2 \vec{e}_1^{\top}\vec{D} \vec{e}_1 &=  -\eta\alpha \vec{e}_1^{\top} \vec{e}_1   
	     \\
	     \implies \alpha &=0, \text{ or, } \vec{y} = \vec{0}.
    %\end{aligned}
    \end{align}
	\end{proof}
	\item
			\label{corr:foclen}
	 The {\em focal length} of the standard parabola, , defined to be the distance between the vertex and the focus, measured along the axis of symmetry, is $\abs{\frac{\eta}{4 \lambda_2}}$
	
	 \end{enumerate}
