\iffalse
\documentclass[12pt]{article}
\usepackage{graphicx}
%\documentclass[journal,12pt,twocolumn]{IEEEtran}
\usepackage[none]{hyphenat}
\usepackage{graphicx}
\usepackage{listings}
\usepackage[english]{babel}
\usepackage{graphicx}
\usepackage{caption}
\usepackage[parfill]{parskip}
\usepackage{hyperref}
\usepackage{booktabs}
\usepackage{gensymb}
%\usepackage{setspace}\doublespacing\pagestyle{plain}
\def\inputGnumericTable{}
\usepackage{color}                                            %%
    \usepackage{array}                                            %%
    \usepackage{longtable}                                        %%
    \usepackage{calc}                                             %%
    \usepackage{multirow}                                         %%
    \usepackage{hhline}                                           %%
    \usepackage{ifthen}
\usepackage{array}
\usepackage{amsmath}   % for having text in math mode
\usepackage{parallel,enumitem}
\usepackage{listings}
\lstset{
language=tex,
frame=single,
breaklines=true
}
 
%Following 2 lines were added to remove the blank page at the beginning
\usepackage{atbegshi}% http://ctan.org/pkg/atbegshi
\AtBeginDocument{\AtBeginShipoutNext{\AtBeginShipoutDiscard}}
%
%New macro definitions
\newcommand{\mydet}[1]{\ensuremath{\begin{vmatrix}#1\end{vmatrix}}}
\providecommand{\brak}[1]{\ensuremath{\left(#1\right)}}
\providecommand{\norm}[1]{\left\lVert#1\right\rVert}
\newcommand{\solution}{\noindent \textbf{Solution: }}
\newcommand{\myvec}[1]{\ensuremath{\begin{pmatrix}#1\end{pmatrix}}}
\let\vec\mathbf
\begin{document}
\begin{center}
\enlargethispage{-4cm}
\title{\textbf{Vector Algebra}}
\date{\vspace{-5ex}} %Not to print date automatically
\maketitle
\end{center}
\setcounter{page}{1}
\section*{12$^{th}$ Maths - Chapter 10}
This is Problem-3 from Exercise 10.4
\begin{enumerate}
\item If unit vector $\overrightarrow{a}$ makes angles $\frac{\pi}{3}$ with $\hat{i}$, $\frac{\pi}{4}$ with $\hat{j}$ and an acute angle $\theta$ with $\hat{k}$, then find $\theta$ and hence, the components of $\overrightarrow{a}$.

\solution
\fi
		Let 
		\begin{align}
			\vec{A}=\myvec{\cos\theta_1\\\cos\theta_2\\\cos\theta_3}
		\end{align}
		where
		\begin{align}
		\cos\theta_1 &=\cos\frac{\pi}{3}
			=\frac{1}{2}\\
			\cos\theta_2 &=\cos\frac{\pi}{4}\\
			=\frac{1}{\sqrt{2}}
		\end{align}
		Since
\begin{align}
    \norm{\vec{A}}&=1,
\sqrt{\cos^2\theta_1+\cos^2\theta_2+\cos^2\theta_3}&=1
    \implies\sqrt{\frac{1}{2}^2+\frac{1}{\sqrt{2}}^2+\cos^2\theta_3 }&=1\\
    \implies\cos\theta_3 &=\pm\frac{1}{2}
\end{align}
Since $\theta_3$ is an acute angle
\begin{align}
 \cos\theta_3=\frac{1}{2}
\end{align}
    Hence 
\begin{align}
		\vec{A}=\myvec{\frac{1}{2}\\[2pt] \frac{1}{\sqrt{2}}\\[2pt] \frac{1}{2}}
\end{align}
