\begin{enumerate}[label=\thesection.\arabic*,ref=\thesection.\theenumi]
 \item Find the equation of the circle which touches both the axes in first quadrant and whose radius is $a$.
 \item Find the equation of the circle which touches x-axis and whose centre is $(1,2)$
 \item lf the lines $3x-4y+4=0$ and $6x-8y-7=0$ are tangents to a circle, then find the radius of the circle.
 [Hint:Distance between given parallel lines gives the diameter of the circle.]
 \item Find the equation of a circle which touches  both the axes and the line $3x-4y+8=0$ and lies in the third quadrant.
 [Hint:Let $a$ be the radius of the circle, then $(-a,-a)$ will be centre and perpendicular distance from the centre to the given line gives the radius of the circle.]
 \item If the line $y=\sqrt{3}x+K$ touches the circle $x^2=16y,$ then find the value of $K$.
 [Hint:Equate perpendicular distance from the centre of the cirle to its radius].
\item If the line y=mx+1 is tangent to the parabolay $y^2=4x$ then find the value of m.
[Hint:solving the equation of line  and parbola, we obtain a quadratic . equation and then apply the tangency condition giving the value of m]
\item Find the condition that the curves $2x=y^2$ and $2xy=k$ intersect orthogonally.
\item Prove that the curves $xy=4$ and $x^2+y^2=8$ touch each other.
\item Find the co-ordinates of the point on the curve $\sqrt{x}+\sqrt{y}$=4 ot which tangent is equally inclined to the axes.
\item Find the angle of intersection of the curves $y=4-x^2$ and $y=x^2$
\item Prove that the curves $y^2=4x$ and $x^2+y^2-6x+1=0$ touch each each other ot the point (1,2).
\item Find the equation of the normal lines to the curve $3x^2-y^2=8$ which are parallel to the line $x+3y=4$.
\item At what points on the curve $x^2+y^2-2x-4y+1=0$, the tangents are parallel to the y-axis?
\item Show that the line $\frac{x}{a}+\frac{y}{b}=1$ touches the curve $y=be-\frac{x}{a}$ at the point where the curve intersects the axis of $y$.
 \item The equation of the normal to the curve $3x^2-y^2 =8$ which is parallel to the line $x+3y=8$ is
 \begin{enumerate}
 \item $3x-y=8$
 \item $3x+y+8=0$
 \item $x+3y+8=0$
 \item $x+3y=0$
 \end{enumerate}
\item The equation of the tangent to the curve $(1+x^2) =2-x$, where it crosses x-axis
\begin{enumerate}
\item $x+5y=2$
\item $x-5y=2$
\item $5x-y=2$
\item $5x+y=2$
\end{enumerate}
\end{enumerate}
State whether the statements are True or False 
\begin{enumerate}[label=\thesection.\arabic*,ref=\thesection.\theenumi,resume*]
\item State whether the statements in each of the exercis from 33 to 40 are Trueor False justify
\item The shortest distance from the point (2,7) to the circle $x^2+y^2$- 14x-10y-151=0 is equal to 5.
[Hint:The shortest distance is equal to the difference of the redius and the distance between  the  cntre and the given point.]
\item lf the line lx+my=1 is a tangent to the circle $x^2+y^2=a^2$, then the ponit (1,m) lies an a circle.
[Hint:use that distance from tne centre of the centre of the circle to the given line is equal to radius of the circle.]
\item The line 1x+my+n=0 will touch the parabola $y^2=4 ax$ if $ln =am^2$,
\item The line 2x+3y=12 touches the ellipse $\frac{x^2}{9}+\frac{y^2}{4}=2$ at the point (3,2).
\end{enumerate}
