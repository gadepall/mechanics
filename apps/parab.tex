
  \label{app:parab}
	Using 
\eqref{eq:conic_affine}
%such that 
\eqref{eq:conic_quad_form} can be expressed as

%\item  
%Substituting \eqref{eq:conic_affine} in \eqref{eq:conic_quad_form}
\begin{align}
\brak{\vec{P}\vec{y}+\vec{c}}^{\top}\vec{V}\brak{\vec{P}\vec{y}+\vec{c}}+2\vec{u}^{\top}\brak{\vec{P}\vec{y}+\vec{c}}+ f
	= 0, 
\end{align}
yielding 
\begin{align}
\vec{y}^{\top}\vec{P}^{\top}\vec{V}\vec{P}\vec{y}+2\brak{\vec{V}\vec{c}+\vec{u}}^{\top}\vec{P}\vec{y}
+  \vec{c}^{\top}\vec{V}\vec{c} + 2\vec{u}^{\top}\vec{c} + f= 0
\label{eq:conic_simp_one}
\end{align}
%
From \eqref{eq:conic_simp_one} and \eqref{eq:conic_parmas_eig_def},
\begin{align}
\vec{y}^{\top}\vec{D}\vec{y}+2\brak{\vec{V}\vec{c}+\vec{u}}^{\top}\vec{P}\vec{y}
+  \vec{c}^{\top}\brak{\vec{V}\vec{c} + \vec{u}}+ \vec{u}^{\top}\vec{c} + f= 0
\label{eq:conic_simp}
\end{align}
When $\vec{V}^{-1}$ exists, choosing
\begin{align}
%\begin{split}
\vec{V}\vec{c}+\vec{u} &= \vec{0}, \quad \text{or}, \vec{c} = -\vec{V}^{-1}\vec{u},
\label{eq:conic_parmas_c_def}
\end{align}
%
%%From \eqref{eq:conic_parmas_k_def} and 
%%
and substituting \eqref{eq:conic_parmas_c_def}
in \eqref{eq:conic_simp}
yields \eqref{eq:conic_simp_temp_nonparab}. 
	\subsection{}
%\item  
When $\abs{\vec{V}} = 0, \lambda_1 = 0$ and 
\begin{align}
\vec{V}\vec{p}_1 = 0, 
\vec{V}\vec{p}_2 = \lambda_2\vec{p}_2.
\label{eq:conic_parab_eig_prop} 
\end{align}
where $\vec{p}_1,\vec{p}_2$ are the eigenvectors of $\vec{V}$ such that  \eqref{eq:conic_parmas_eig_def}
%
\begin{align}
\vec{P} = \myvec{\vec{p}_1 & \vec{p}_2},
\label{eq:eig_matrix}
\end{align}
Substituting \eqref{eq:eig_matrix}
in \eqref{eq:conic_simp},
\begin{align}
	\vec{y}^{\top}\vec{D}\vec{y}+2\brak{\vec{c}^{\top}\vec{V}+\vec{u}^{\top}}\myvec{\vec{p}_1 & \vec{p}_2}\vec{y}
	+  \vec{c}^{\top}\brak{\vec{V}\vec{c} + \vec{u}}+ \vec{u}^{\top}\vec{c} + f&= 0
\\
\implies \vec{y}^{\top}\vec{D}\vec{y}
+2\myvec{\brak{\vec{c}^{\top}\vec{V}+\vec{u}^{\top}}\vec{p}_1  \brak{\vec{c}^{\top}\vec{V}+\vec{u}^{\top}}\vec{p}_2}\vec{y}
	+  \vec{c}^{\top}\brak{\vec{V}\vec{c} + \vec{u}}+ \vec{u}^{\top}\vec{c} + f&= 0
\\
\implies \vec{y}^{\top}\vec{D}\vec{y}
+2\myvec{\vec{u}^{\top}\vec{p}_1 & \brak{\lambda_2\vec{c}^{\top}+\vec{u}^{\top}}\vec{p}_2}\vec{y}
	+  \vec{c}^{\top}\brak{\vec{V}\vec{c} + \vec{u}}+ \vec{u}^{\top}\vec{c} + f&= 0
\end{align}
upon substituting from 
 \eqref{eq:conic_parab_eig_prop} yielding
\begin{align}
\lambda_2y_2^2+2\brak{\vec{u}^{\top}\vec{p}_1}y_1+  2y_2\brak{\lambda_2\vec{c}+\vec{u}}^{\top}\vec{p}_2
	+  \vec{c}^{\top}\brak{\vec{V}\vec{c} + \vec{u}}+ \vec{u}^{\top}\vec{c} + f= 0
\label{eq:conic_parab_foc_len_temp} 
\end{align}
%which is the equation of a parabola. 
Thus, \eqref{eq:conic_parab_foc_len_temp} 
can be expressed as \eqref{eq:conic_simp_temp_parab} by choosing
\begin{align}
%\label{eq:eta}
\eta = 2\vec{u}^{\top}\vec{p}_1
\end{align}
%Choosing 
%\begin{align}
%\vec{u} + \lambda_2\vec{c} = 0,
%\vec{c}^{\top}\brak{\vec{V}\vec{c} + \vec{u}}+ \vec{u}^{\top}\vec{c} + f = 0,
%\end{align}
% the above equation becomes
%\begin{align}
%y_2^2= -\frac{2\vec{u}^{\top}\vec{p}_1}{ \lambda_2} \brak{y_1
%+  \frac{\vec{u}^{\top}\vec{V}\vec{u} - 2\lambda_2\vec{u}^{\top}\vec{u} + f\lambda_2^2}{2\vec{u}^{\top}\vec{p}_1\lambda_2^2}}
%\\
%or \eta = 2\vec{u}^{\top}\vec{p}_1
%%\label{eq:conic_simp_parab_new}
%\end{align}
and $\vec{c}$ in \eqref{eq:conic_simp} such that
\begin{align}
\label{eq:conic_parab_one}
2\vec{P}^{\top}\brak{\vec{V}\vec{c}+\vec{u}} &= \eta\myvec{1\\0}
\\
\vec{c}^{\top}\brak{\vec{V}\vec{c} + \vec{u}}+ \vec{u}^{\top}\vec{c} + f&= 0
\label{eq:conic_parab_two}
\end{align}
%we obtain  \eqref{eq:conic_simp_temp_parab}.
$\because
\vec{P}^{\top}\vec{P} = \vec{I}$,
multiplying \eqref{eq:conic_parab_one} by $\vec{P}$ yields
\begin{align}
\label{eq:conic_parab_one_eig}
	\brak{\vec{V}\vec{c}+\vec{u}} &= \frac{\eta}{2}\vec{p}_1,
\end{align}
which, upon substituting in \eqref{eq:conic_parab_two}
results in 
\begin{align}
\frac{\eta}{2}\vec{c}^{\top}\vec{p}_1 + \vec{u}^{\top}\vec{c} + f&= 0
\label{eq:conic_parab_two_eig}
\end{align}
\eqref{eq:conic_parab_one_eig} and \eqref{eq:conic_parab_two_eig} can be clubbed together to obtain \eqref{eq:conic_parab_c}.
